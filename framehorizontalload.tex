\pgfkeys{
/framehorizontalload/.is family, /framehorizontalload,
default/.style={
startx = 0cm,
starty = 0cm,
endx = 5cm,
endy = 5cm,
number of arrows = 10,
load direction = 1,
space between arrows = 0.02cm,
space = 0.1cm,
start ratio = 0.02,
end ratio = 0.02,
load text = $\SI{10}{\kilo\newton/\meter}$,
text color = black,
text shiftx=0pt,
text shifty=0pt,
arrow length=5pt,
arrow width=3pt,
arrow scale=1,
arrow line thickness=1pt
arrow color = black,},
startx/.store in = \loadstartcoordx,
starty/.store in = \loadstartcoordy,
endx/.store in = \loadendcoordx,
endy/.store in = \loadendcoordy,
number of arrows/.store in = \loadarrownumber,
load direction/.store in = \loaddir,
space between arrows/.store in = \spbtwarr,
space/.store in = \loadarrowspace,
start ratio/.store in = \loadstartratio,
end ratio/.store in = \loadendratio,
load text/.store in = \loadtext,
text color/.store in = \textcolor,
text shiftx/.store in = \textshiftx,
text shifty/.store in = \textshifty,
arrow length/.store in = \arrlen,
arrow width/.store in = \arrwid,
arrow scale/.store in = \arrscale,
arrow line thickness/.store in = \arrlinethk,
arrow color/.store in = \arrcolor,}
\newcommand{\framehorizontalload}[1][]{
\pgfkeys{/framehorizontalload, default, #1}
\tikzmath{
real \loadstartcoordx, \loadstartcoordy, \loadendcoordx, \loadendcoordy;
real \loadangle, \loadunx, \loaduny, \loaddxf, \loaddyf;
real \loaddxa, \loaddya, \loadaaa, \loadbbb;
real \loadxvalue, \loadyvalue;
real \loadwxvalue, \loadwyvalue;
real \loadstartratio, \loadendratio, \lenf;
int \loadarrownumber, \loadarrownumbermo, \i, \j, \loaddir;
real \arrlen, \arrwid, \arrscale, \arrlinethk;
%Conversion all units to the unit pt
\loadstartcoordx = \loadstartcoordx;
\loadstartcoordy = \loadstartcoordy;
\loadendcoordx = \loadendcoordx;
\loadendcoordy = \loadendcoordy;
\loadarrowspace = \loadarrowspace;
\arrlen=\arrlen;
\arrwid=\arrwid;
\arrlinethk=\arrlinethk;
%End conversion, All values are now in pt units.
%% Force application line
\loaddxf = (\loadendcoordx-\loadstartcoordx); %dx for the whole frame
\loaddyf = (\loadendcoordy-\loadstartcoordy); %dy for the whole frame
\loadangle = atan2(\loaddyf,\loaddxf);
%% Start of arrows
\arrforceratio= 1-(\loadstartratio+\loadendratio);
\loaddxa = \loaddxf*(\arrforceratio); %dx for the start and end of arrow
\loaddya = \loaddyf*(\arrforceratio); %dy for the start and end of arrow
%%%
\loadwxvalue = (\loaddxa)/(\loadarrownumber-1); %dx for the single arrow
\loadwyvalue = (\loaddya)/(\loadarrownumber-1); %dx for the single arrow
\horloadlen= veclen(\loadwxvalue, \loadwyvalue) - \spbtwarr;
%%
\loadxvalue{1} = \loadstartcoordx + \loaddxf*(\loadstartratio);
\loadyvalue{1} = \loadstartcoordy + \loaddyf*(\loadstartratio);
%%
\loadarrownumbermo = \loadarrownumber-1;
%% Mid Region
for \i in {1,...,\loadarrownumber}{
\loadxvalue{\i} = \loadxvalue{1} + (\i-1)*\loadwxvalue;
\loadyvalue{\i} = \loadyvalue{1} + (\i-1)*\loadwyvalue;
};
%%%
\arrspang=90+\loadangle;
%%
\horzarrdir = ifthenelse(\loaddir==0,"arrforce","forcearr");
%%
\texti=int(\loadarrownumber/2);
\textshifty=\textshifty+\loadarrowspace;
}

\tikzset{
forcearr/.style={
arrows={<[length=\arrlen, width=\arrwid, scale=\arrscale, flex=1, bend]-}}
};

\tikzset{
arrforce/.style={
arrows={->[length=\arrlen, width=\arrwid, scale=\arrscale, flex=1, bend]}}
};

\tikzset{
forcearrarr/.style={
arrows={<[length=\arrlen, width=\arrwid, scale=\arrscale, flex=1, bend]-
>[length=\arrlen, width=\arrwid, scale=\arrscale, flex=1, bend]}}
};

\tikzset{
loadarrowangle/.style={
arrows={->[flex=1, bend, width=3pt, length=4pt]}}
};

\begin{scope}[x=1pt, y=1pt]; % Drawing everything in pt units

% Inner Arrows
\foreach \i in {1,...,\loadarrownumbermo}{
\draw [\horzarrdir, line width=\arrlinethk, \arrcolor]
(\loadxvalue{\i}, \loadyvalue{\i})
++(\arrspang:\loadarrowspace) --
++(\loadangle:\horloadlen);}

\path (\loadstartcoordx, \loadstartcoordy) --
node[sloped, above, xshift=\textshiftx, yshift=\textshifty]{\loadtext}(\loadendcoordx, \loadendcoordy);
\end{scope}
}