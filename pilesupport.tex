\pgfkeys{
 /pilesupport/.is family, /pilesupport,
  default/.style = {
  x coordinate=0cm,
  y coordinate=0cm,  
  pile depth=4cm,
  pile diameter=0.5cm,
  pile line thickness=1pt,
  fill color=gray,
  draw boundary line=0,
  boundary line thickness=2pt,
  boundary line color=green,
  boundary line type=dashed,
  horizontal show springs=1,
  horizontal number of springs = 5,
  horizontal spring direction = 1,
  horizontal space between springs = 0.15cm,
  horizontal space = 0.25cm,
  horizontal start ratio = 0.05,
  horizontal end ratio = 0.05,
  horizontal spring text = {},
  horizontal text color = black,
  horizontal text shiftx=0pt,
  horizontal text shifty=0pt,
  horizontal spring prelength ratio=0.2,
  horizontal spring postlength ratio=0.2,
  horizontal spring segment=0.15cm,
  horizontal spring width=3pt,
  horizontal spring scale=1,
  horizontal spring line thickness=1pt,
  horizontal spring color = black,
  horizontal support width=0.3cm,
  horizontal support depth=0.1cm,
  horizontal support line thickness=1pt,
  horizontal show support shade=1,
  horizontal support shade color=gray,
  vertical show springs=1,
  vertical number of springs = 5,
  vertical space = 0.0cm,
  vertical start ratio = 0.1,
  vertical end ratio = 0.1,
  vertical spring text = {},
  vertical text shiftx=0pt,
  vertical text shifty=0pt,
  vertical spring length=0.5cm,
  vertical spring prelength ratio=0.15,
  vertical spring postlength ratio=0.15,
  vertical spring width=0.1cm,
  vertical spring segment=0.15cm,
  vertical spring scale=1,
  vertical spring line thickness=1pt,
  vertical spring color=black,
  vertical support width=0.3cm,
  vertical support depth = 0.1cm,
  vertical support line thickness = 1pt,
  vertical show support shade=1,
  vertical support shade color=gray,
  axial show spring=1,
  axial spring length=0.5cm,
  axial spring prelenratio=0.25,
  axial spring postlength ratio=0.1,
  axial spring segment=0.15cm,
  axial spring width=0.1cm,
  axial spring line thickness=1pt,
  axial spring color=black,
  axial support width=0.3cm,
  axial support depth = 0.1cm,
  axial support line thickness = 1pt,
  axial show support shade=1,
  axial support shade color=gray,},
  x coordinate/.store in=\coordinatex,
  y coordinate/.store in=\coordinatey,
  pile depth/.store in=\pdepth,
  pile diameter/.store in=\pdiameter,
  pile line thickness/.store in=\plinethick,  
  fill color/.store in=\fillcolor,
  draw boundary line/.store in=\drawboundryline,
  boundary line thickness/.store in=\blinethick,
  boundary line color/.store in=\blinecolor,
  boundary line type/.store in=\blinetype,
  horizontal show springs/.store in = \showhorizontalsprings,
  horizontal number of springs/.store in = \horspringnumber,
  horizontal spring direction/.store in = \horspringdir,
  horizontal space between springs/.store in = \horspbtwspr,
  horizontal space/.store in = \horspringspace,
  horizontal start ratio/.store in = \horspringstartratio,
  horizontal end ratio/.store in = \horspringendratio,
  horizontal spring text/.store in = \horspringtext,
  horizontal text color/.store in = \hortextcolor,
  horizontal text shiftx/.store in = \hortextshiftx,
  horizontal text shifty/.store in = \hortextshifty,
  horizontal spring prelength ratio/.store in = \horprelenratio,
  horizontal spring postlength ratio/.store in = \horpostlenratio,
  horizontal spring segment/.store in = \horsegm,
  horizontal spring width/.store in = \horsprwid,
  horizontal spring scale/.store in = \horsprscale,
  horizontal spring line thickness/.store in = \horsprlinethk,
  horizontal spring color/.store in = \horsprcolor,
  horizontal support width/.store in = \horsuppwidth,
  horizontal support depth/.store in = \horsuppdepth,
  horizontal support line thickness/.store in = \horsupplinethk,
  horizontal show support shade/.store in = \horshowsuppshade,
  horizontal support shade color/.store in = \horsuppshadecol,
  vertical show springs/.store in = \showverticalsprings,
  vertical number of springs/.store in = \verspringnumber,
  vertical space/.store in = \verspringspace,
  vertical start ratio/.store in = \verspringstartratio,
  vertical end ratio/.store in = \verspringendratio,
  vertical spring text/.store in = \verspringtext,
  vertical text shiftx/.store in = \vertextshiftx,
  vertical text shifty/.store in = \vertextshifty,
  vertical spring length/.store in = \verspringlength,
  vertical spring prelength ratio/.store in = \verprelenratio,
  vertical spring postlength ratio/.store in = \verpostlenratio,
  vertical spring width/.store in = \verampl,
  vertical spring segment/.store in = \versegm,
  vertical spring scale/.store in = \verspringscale,
  vertical spring line thickness/.store in = \verspringthk,
  vertical spring color/.store in = \verspringcolor,
  vertical support width/.store in = \versuppwidth,
  vertical support depth/.store in = \versuppdepth,
  vertical support line thickness/.store in = \versupplinethk,
  vertical show support shade/.store in = \vershowsuppshade,
  vertical support shade color/.store in = \versuppshadecol,
  axial show spring/.store in = \showaxialspring,
  axial spring length/.store in=\axialspringlength,
  axial spring prelenratio/.store in=\axialspringprelenratio,
  axial spring postlength ratio/.store in=\axialspringpostlenratio,
  axial spring segment/.store in=\axialspringsegm,
  axial spring width/.store in=\axialspringwidth,
  axial spring line thickness/.store in=\axialspringlinethk,
  axial spring color/.store in=\axialspringcolor,
  axial support width/.store in = \axialsuppwidth,
  axial support depth/.store in = \axialsuppdepth,
  axial support line thickness/.store in = \axialsupplinethk,
  axial show support shade/.store in = \axialshowsuppshade,
  axial support shade color/.store in = \axialsuppshadecol,}

\newcommand{\pilesupport}[1][]{
\pgfkeys{/pilesupport, default, #1}
\tikzmath{
real \coordinatex, \coordinatey, \pdepth, \pdiameter, \plinethick;
real \blinethick, \blinestartx, \blinewidth;
int \drawboundryline;
int \showhorizontalsprings, \horspringnumber, \horspringdir,
real \horspbtwspr, \horspringspace, \horspringstartratio, \horspringendratio;
real \hortextshiftx, \hortextshifty, \horprelenratio, \horpostlenratio;
real \horsegm, \horsprwid, \horsprscale, \horsprlinethk;
real \horsupportdist, \horsupportdistdx, \horsupportdistdy, \horsupportendx, \horsupportendy;
real \horsuppwidth, \horsupdepth, \horsupplinethk, \horsuppangle;
int \horshowsuppshade;
int \showverticalsprings, \verspringnumber;
real \verspringspace, \verspringstartratio, \verspringendratio;
real \vertextshiftx, \vertextshifty, \verspringlength;
real \verprelenratio, \verpostlenratio, \verampl, \versegm;
real \verspringscale, \verspringthk;
real \versuppwidth, \versuppdepth, \versupplinethk;
int \vershowsuppshade;
int \showaxialspring;
real \axialspringlength, \axialspringprelenratio, \axialspringpostlenratio;
real \axialspringsegm, \axialspringwidth, \axialspringlinethk,
real \axialspringprelength, \axialspringpostlength, \axialspringx, \axialspringy;
real \axialsuppwidth, \axialsuppdepth, \axialsupplinethk;
int \axialshowsuppshade;
%Conversion all units to the unit pt
\coordinatex=\coordinatex;
\coordinatey=\coordinatey;
\pdepth=\pdepth;
\pdiameter=\pdiameter;
\plinethick=\plinethick;
\blinethick=\blinethick;
\verspringspace=\verspringspace;
\vertextshiftx=\vertextshiftx;
\vertextshifty=\vertextshifty;
\verspringlength=\verspringlength;
\verampl=\verampl;
\versegm=\versegm;
\verspringthk=\verspringthk;
\versuppwidth=\versuppwidth;
\versuppdepth=\versuppdepth;
\versupplinethk=\versupplinethk;
\hortextshiftx=\hortextshiftx;
\hortextshifty=\hortextshifty;
\horsprwid=\horsprwid;
\horsegm=\horsegm;
\horsprlinethk=\horsprlinethk;
\horsuppwidth=\horsuppwidth;
\horsuppdepth=\horsuppdepth;
\horsupplinethk=\horsupplinethk;
\axialspringlength=\axialspringlength;
\axialspringsegm=\axialspringsegm;
\axialspringwidth=\axialspringwidth;
\axialspringlinethk=\axialspringlinethk;
\axialsuppwidth=\axialsuppwidth;
\axialsuppdepth=\axialsuppdepth;
\axialsupplinethk=\axialsupplinethk;
%End conversion, All values are now in pt units.
\blinestartx=\pdiameter/2+\plinethick/2+\blinethick/2;
\blinewidth=2*\blinestartx;
%%
\axialspringprelength=\axialspringprelenratio*\axialspringlength;
\axialspringpostlength=\axialspringpostlenratio*\axialspringlength;
\axialspringx=0;
\axialspringy=-\pdepth;
\axialsupporty=-\pdepth-\axialspringlength;
}



\begin{scope}[x=1pt, y=1pt, xshift=\coordinatex, yshift=\coordinatey, rotate=0]; % Drawing everything in pt units

\fill[fill=gray](-\pdiameter/2, 0) rectangle (+\pdiameter/2, -\pdepth);
\draw[line width=\plinethick]
(-\pdiameter/2, -\pdepth) -- ++(0,\pdepth) -- ++(\pdiameter,0) -- ++(0,-\pdepth);
 
\filldraw[line width=\plinethick, fill=white]
(-\pdiameter/4, -\pdepth) ellipse[x radius=\pdiameter/4, y radius=\pdiameter/8];
 
\fill[fill=gray]
(0, -\pdepth) arc[start angle=180, end angle=360, x radius=\pdiameter/4, y radius=\pdiameter/8];
 
 \draw[line width=\plinethick]
(0, -\pdepth) arc[start angle=180, end angle=360, x radius=\pdiameter/4, y radius=\pdiameter/8];
 
%% Boundary Line
\ifthenelse{\drawboundryline=1}{
\draw[line width=\blinethick, \blinecolor, \blinetype]
(-\blinestartx,0) -- ++(0,-\pdepth) -- ++(\blinewidth,0) -- ++(0,\pdepth);}{}

%% Lateral Springs
\ifthenelse{\showhorizontalsprings=1}{
\framedistributedspring[startx = \pdiameter/2,
starty = 0cm,
endx = \pdiameter/2,
endy = -\pdepth,
number of springs = \verspringnumber,
space = \verspringspace,
start ratio = \verspringstartratio,
end ratio = \verspringendratio,
spring text = {},
text shiftx=\vertextshiftx,
text shifty=\vertextshifty,
spring length=\verspringlength,
spring prelength ratio=\verprelenratio,
spring postlength ratio=\verpostlenratio,
spring width=\verampl,
spring segment=\versegm,
spring scale=\verspringscale,
spring line thickness=\verspringthk,
spring color=\verspringcolor,
support width=\versuppwidth,
support depth = \versuppdepth,
support line thickness = \versupplinethk,
show support shade=\vershowsuppshade,
support shade color=\versuppshadecol,]}

%% Vertical Springs
\ifthenelse{\showverticalsprings=1}{
\framehorizontalspring[startx = -\pdiameter/2,
starty = -\pdepth,
endx = -\pdiameter/2,
endy = 0,
number of springs = \horspringnumber,
spring direction = \horspringdir,
space between springs =\horspbtwspr,
space = \horspringspace,
start ratio = \horspringstartratio,
end ratio = \horspringendratio,
spring text = {},
text color = black,
text shiftx=\hortextshiftx,
text shifty=\hortextshifty,
spring prelength ratio=\horprelenratio,
spring postlength ratio=\horpostlenratio,
spring segment=\horsegm,
spring width=\horsprwid,
spring scale=\horsprscale,
spring line thickness=\horsprlinethk,
spring color=\horsprcolor,
support width=\horsuppwidth,
support depth=\horsuppdepth,
support line thickness=\horsupplinethk,
show support shade=\horshowsuppshade,
support shade color=\horsuppshadecol,]}

%% Axial Springs
\ifthenelse{\showaxialspring=1}{
\draw[line width=\axialspringlinethk, \axialspringcolor, line join=round,
decorate,
decoration={zigzag,
	amplitude=\axialspringwidth,
	pre length=\axialspringprelength,
	post length=\axialspringpostlength,
	segment length=\axialspringsegm,}] 
(\axialspringx, \axialspringy) -- ++(0,-\axialspringlength);

\fixedsupport[%
support width = \axialsuppwidth,
support depth = \axialsuppdepth,
line thickness = \axialsupplinethk,
x value = 0,
y value = \axialsupporty,
rotation = 0,
show shade=\axialshowsuppshade,
shade color=\axialsuppshadecol,]}
 
 
\end{scope}
}

