\pgfkeys{
 /pilesupport/.is family, /pilesupport,
  default/.style = {
  x coordinate=0cm,
  y coordinate=0cm,  
  pile depth=4cm,
  pile diameter=0.5cm,
  pile line thickness=1pt,
  fill color=gray,
  draw boundary line=0,
  boundary line thickness=2pt,
  boundary line color=green,
  boundary line type=dashed,},
  x coordinate/.store in=\coordinatex,
  y coordinate/.store in=\coordinatey,
  pile depth/.store in=\pdepth,
  pile diameter/.store in=\pdiameter,
  pile line thickness/.store in=\plinethick,  
  fill color/.store in=\fillcolor,
  draw boundary line/.store in=\drawboundryline,
  boundary line thickness/.store in=\blinethick,
  boundary line color/.store in=\blinecolor,
  boundary line type/.store in=\blinetype,
}

\newcommand{\pilesupport}[1][]{
\pgfkeys{/pilesupport, default, #1}
\tikzmath{
real \coordinatex, \coordinatey, \pdepth, \pdiameter, \plinethick;
real \blinethick, \blinestartx, \blinewidth;
int \drawboundryline;
%Conversion all units to the unit pt
\coordinatex=\coordinatex;
\coordinatey=\coordinatey;
\pdepth=\pdepth;
\pdiameter=\pdiameter;
\plinethick=\plinethick;
%\blinethick=5pt;
%\drawboundryline=1;
\blinethick=\blinethick;
%End conversion, All values are now in pt units.
\blinestartx=\pdiameter/2+\plinethick/2+\blinethick/2;
\blinewidth=2*\blinestartx;
}
\begin{scope}[x=1pt, y=1pt, xshift=\coordinatex, yshift=\coordinatey, rotate=0]; % Drawing everything in pt units

\fill[fill=gray](-\pdiameter/2, 0) rectangle (+\pdiameter/2, -\pdepth);
 \draw[line width=\plinethick]
 (-\pdiameter/2, -\pdepth) -- ++(0,\pdepth) -- ++(\pdiameter,0) -- ++(0,-\pdepth);
 
 \filldraw[line width=\plinethick, fill=white]
 (-\pdiameter/4, -\pdepth) ellipse[x radius=\pdiameter/4, y radius=\pdiameter/8];
 
 \fill[fill=gray]
 (0, -\pdepth) arc[start angle=180, end angle=360, x radius=\pdiameter/4, y radius=\pdiameter/8];
 
  \draw[line width=\plinethick]
 (0, -\pdepth) arc[start angle=180, end angle=360, x radius=\pdiameter/4, y radius=\pdiameter/8];
 
 %% Boundary Line
 \ifthenelse{\drawboundryline=1}{
 \draw[line width=\blinethick, \blinecolor, \blinetype]
 (-\blinestartx,0) -- ++(0,-\pdepth) -- ++(\blinewidth,0) -- ++(0,\pdepth);}{}
 
\end{scope}
}

