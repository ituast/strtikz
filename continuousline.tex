\pgfkeys{
 /continuousline/.is family, /continuousline,
  default/.style={
  startx = 0cm,
  starty = 0cm,
  endx = 4cm,
  endy = 5cm,
  sign radius = 0.2cm,
  sign curvature ratio = 0.4,
  sign bump ratio = 2,
  sign location ratio = 0.5,
  line thickness = 2pt,
  line color = black,},
  startx/.store in =\startx,
  starty/.store in =\starty,
  endx/.store in =\endx,
  endy/.store in =\endy,
  sign radius/.store in =\signrad,
  sign curvature ratio/.store in =\signcurvratio,
  sign bump ratio/.store in =\signbumpratio,
  sign location ratio/.store in =\signlocratio,
  line thickness/.store in =\linewid,
  line color/.store in =\lcolor,
}


\newcommand{\continuousline}[1][]{
\pgfkeys{/continuousline, default, #1}

\tikzmath{
real \startx, \starty, \endx, \endy;
real \linewid;
real \signrad, \signcurvratio, \signlocratio;
real \linelength, \halflen, \signlena, \signlenb;
real \lenx, \leny;
real \midx, \midy;
real \gam, \alp, \bet, \curvang, \curvlen;
%Conversion all units to the unit pt
\startx = \startx;
\starty = \starty;
\endx = \endx;
\endy = \endy;
\linewid = \linewid;
\signrad = \signrad;
%
\lenx = (\endx - \startx);
\leny = (\endy - \starty);
\linelength = veclen(\lenx, \leny);
\alp = atan2(\leny,\lenx);
\halflen = \linelength*\signlocratio;
\signlena = \halflen - 2*\signrad;
\signlenb = \halflen + 2*\signrad;
\midx{1} = \startx + \lenx*(\signlena/(\linelength));
\midy{1} = \starty + \leny*(\signlena/(\linelength));
\midx{2} = \startx + \lenx*\signlocratio;
\midy{2} = \starty + \leny*\signlocratio;
\midx{3} = \startx + \lenx*(\signlenb/(\linelength));
\midy{3} = \starty + \leny*(\signlenb/(\linelength));
\gam = atan(\signcurvratio);
\curvang{1} = \alp+90-\gam;
\curvang{2} = \alp+90+\gam;
\curvang{3} = \curvang{2}+180;
\curvang{4} = \curvang{1}+180;
\curvlen = \signrad/(cos(\gam));
\newcurvlen=\signbumpratio*\curvlen;
}


\begin{scope}[x=1pt, y=1pt];    % Draw everything in pt units

\draw[line width=\linewid, \lcolor] (\startx,\starty) -- (\midx{1}, \midy{1})
.. controls ++(\curvang{1}:\newcurvlen) and ++(\curvang{2}:\newcurvlen) ..
(\midx{2}, \midy{2})
.. controls ++(\curvang{3}:\newcurvlen) and ++(\curvang{4}:\newcurvlen) ..
(\midx{3}, \midy{3}) -- (\endx,\endy);


\end{scope}
}
