\pgfkeys{
 /framestructure/.is family, /framestructure,
  default/.style = {
  number of storys = 5,
  number of bays = 5,
  show less bays=0,
  left bays=1,
  right bays=1,
  story height = 1cm,
  bay width = 2cm,
  startX = 0cm,
  startY = 0cm,
  line thickness = 1.5pt,
  beam line thickness = 1.5pt,
  ground beam line thickness = 1.5pt,
  beam color=blue,
  column line thickness = 1.5pt,
  column color=black,
  support width = 0.3cm,
  support height = 0.15cm,
  support line thickness = 1.5pt,
  show supports = 1,
  number of isolators = 4,
  isolator width = 0.3cm,
  isolator thickness = 0.2cm,
  isolator line thickness = 1.5pt,
  isolator shift = 0,
  isolator shift distance = 0.2cm,
  foundation thickness = 0.5cm,
  foundation side width = 1cm,
  show mass = 1,
  mass radius = 2pt,
  mass color super=black,
  mass color sub=red,
  show dof = 1,
  show dofx = 1,
  show dofy = 1,
  show dofr = 1,
  dof xstring = $x$,
  dof ystring = $y$,  
  dof rstring = $\theta$,
  dof floor = 1,
  dof column = 2,
  arrow length ratio = 0.4,
  rot arrow length ratio = 0.4,
  dof x rotation = 0,
  dof y rotation = 0,
  dof r rotation = 0,
  dof offset ratio = 0.075,
  rotation dof start angle = 120,
  rotation dof end angle = 280,
  dofx pos = 1,
  dofy pos = 1,
  dofr pos = 1,
  dofx in sep=1pt,
  dofy in sep=1pt,
  dofr in sep=1pt,
  dof offset x= 0.1cm,
  dof offset y= 0.1cm,
  dof offset rot x= 0.1cm,
  dof offset rot y= 0.1cm, 
  dof arrow ratio=0.75,
  dof text ratio=0.5,
  show axes = 1,
  subfloor number=2,
  half space type=1,
  left soil distance = 2cm,
  right soil distance = 4cm,
  left soil depth = 2cm,
  right soil depth = 2cm,
  soil below foundation = 2cm,
  left control distance x = 2cm,
  left control distance y = 2cm,
  right control distance x = 2cm,
  right control distance y = 2cm,
  axis seperation = 0.2cm,
  x axis length = 0.5cm,
  y axis length = 0.5cm,
  show piles=1,
  number of piles=3,
  pile depth=4cm,
  pile side space=0.5cm,
  pile diameter=0.5cm,
  pile line thickness=1pt,
  show pile boundary line=0,
  pile boundary line thickness=2pt,
  pile boundary line color=blue,
  pile boundary line type=dashed,
  show lateral load = 0,
  lateral load shift=0.5cm,
  lateral load type=2,
  top arrow length=1.5cm,
  base arrow length=0.75cm,
  show deflection = 0, 
  interstory drift = 0.2cm,
  defl parameter x = 0.2cm,
  defl parameter y = 0.5cm,
  defl parameter base = 0.15cm,
  show eng bedrock=0,
  eng bedrock depth=2cm,
  eng bedrock left distance=1cm,
  eng bedrock right distance=1cm,
  eng bedrock line thickness=1.5pt,
  eng bedrock text=Bedrock,
  show trans layer=0,
  trans layer line thickness=2pt,
  trans layer color=red,
  trans layer line type=dashed,
  show ss interface=0,
  ss interface line thickness=3pt,
  ss interface color=green,
  ss interface line type=dashed,
  show substructure mark=0,
  substructure mark space=0.30cm,
  substructure mark radius=0.25cm,
  substructure mark line thickness=0.75pt,
  substructure mark line type=dashed,
  substructure mark line color=blue,
  show superstructure shade=0,
  superstructure shade space=5pt,
  superstructure shade opacity=0.75,
  show superstructure mark=0,
  superstructure mark space=0.30cm,
  superstructure mark radius=0.25cm,
  superstructure mark line thickness=0.75pt,
  superstructure mark line type=dashed,
  superstructure mark line color=red,},
  number of storys/.store in = \storynumber,
  number of bays/.store in = \baynumber,
  show less bays/.store in=\showlessbays,
  left bays/.store in=\leftbays,
  right bays/.store in=\rightbays,
  story height/.store in = \storyheight,
  bay width/.store in = \baywidth,
  startX/.store in = \startx,
  startY/.store in = \starty,
  line thickness/.store in = \linet,
  beam line thickness/.store in=\beamlinet,
  ground beam line thickness/.store in=\groundbeamlinet,
  beam color/.store in=\beamcolor,
  column line thickness/.store in=\collinet,
  column color/.store in=\columncolor,
  support width/.store in = \supportwidth,
  support height/.store in = \supportheight,
  support line thickness/.store in = \baselinet,
  show supports/.store in = \showsupports,
  number of isolators/.store in = \numberofisolators,
  isolator width/.store in = \isolationwidth,
  isolator thickness/.store in = \isolationdepth,
  isolator line thickness/.store in = \isolinet,
  isolator shift/.store in = \isoshiftyn,
  isolator shift distance/.store in = \isoshift,
  foundation thickness/.store in = \foundationdepth,
  foundation side width/.store in = \foundsidew,
  show mass/.store in = \showmass,
  mass radius/.store in = \massrad,
  mass color super/.store in = \masscolorsuper,
  mass color sub/.store in = \masscolorsub,
  show dof/.store in = \showdof,
  show dofx/.store in = \shodofx,
  show dofy/.store in = \shodofy,
  show dofr/.store in = \shodofr,
  dof floor/.store in = \doflocfloor,
  dof xstring/.store in = \dofxstr,
  dof ystring/.store in = \dofystr,
  dof rstring/.store in = \dofrstr,
  dof column/.store in = \dofloccolumn,
  arrow length ratio/.store in = \arrowlenratio,
  rot arrow length ratio/.store in = \rotarrowlenratio,
  dof offset ratio/.store in = \dofoffsetratio,
  dof x rotation/.store in = \dofxrotation,
  dof y rotation/.store in = \dofyrotation,
  dof r rotation/.store in = \dofrrotation,
  rotation dof start angle/.store in = \rotdofstartangle,
  rotation dof end angle/.store in = \rotdofendangle,
  dofx pos/.store in = \dofposx,
  dofy pos/.store in = \dofposy,
  dofr pos/.store in = \dofposr,
  dofx in sep/.store in = \dofinnersepx,
  dofy in sep/.store in = \dofinnersepy,
  dofr in sep/.store in = \dofinnersepr,
  dof offset x/.store in = \dofoffsetx,
  dof offset y/.store in = \dofoffsety,
  dof offset rot x/.store in = \dofrotoffsetx,  
  dof offset rot y/.store in = \dofrotoffsety,
  dof arrow ratio/.store in = \dofarrowratio,
  dof text ratio/.store in = \doftextratio,
  show axes/.store in = \showaxes,
  subfloor number/.store in=\subfloors,
  half space type/.store in=\halfspacetype,
  left soil distance/.store in = \leftsoildist,
  right soil distance/.store in = \rightsoildist,
  left soil depth/.store in = \leftsoildepth,
  right soil depth/.store in = \rightsoildepth,
  soil below foundation/.store in = \soilbelowfound,
  left control distance x/.store in = \leftcontrolx,
  left control distance y/.store in = \leftcontroly,
  right control distance x/.store in = \rightcontrolx,
  right control distance y/.store in = \rightcontroly,
  axis seperation/.store in = \axisseperation,
  x axis length/.store in = \axeslenX,
  y axis length/.store in = \axeslenY,
  show piles/.store in=\showpiles,
  number of piles/.store in=\numberofpiles,
  pile depth/.store in=\piledepth,
  pile side space/.store in=\pilesidespace,
  pile diameter/.store in=\pilediameter,
  pile line thickness/.store in=\pilelinethickness,
  show pile boundary line/.store in=\showpbline,
  pile boundary line thickness/.store in=\pblinet,
  pile boundary line color/.store in=\pblinecolor,
  pile boundary line type/.store in=\pblinetype,
  show lateral load/.store in=\showlatload,
  lateral load shift/.store in=\latloadshift,
  lateral load type/.store in=\latloadtype,
  top arrow length/.store in=\toparrlen,
  base arrow length/.store in=\basearrlen,
  show deflection/.store in=\showdefl, 
  interstory drift/.store in=\drift,
  defl parameter x/.store in=\defparx,
  defl parameter y/.store in=\defpary,
  defl parameter base/.store in=\defbase,
  show eng bedrock/.store in=\showengbedrock,
  eng bedrock depth/.store in=\engbedrockdepth,
  eng bedrock left distance/.store in=\engbedrockleftdist,
  eng bedrock right distance/.store in=\engbedrockrightdist,
  eng bedrock line thickness/.store in=\engbedrocklinewidth,
  eng bedrock text/.store in=\engbedrocktext,
  show trans layer/.store in=\showtranslayer,
  trans layer line thickness/.store in= \translayerlinet,
  trans layer color/.store in=\translayercolor,
  trans layer line type/.store in=\translayerlinetype,
  show ss interface/.store in=\showssinter,
  ss interface line thickness/.store in=\ssinterlinet,
  ss interface color/.store in=\ssintercolor,
  ss interface line type/.store in=\ssinterlinetype,
  show substructure mark/.store in=\showmarkss,
  substructure mark space/.store in=\markssexspace,
  substructure mark radius/.store in=\markssrad,
  substructure mark line thickness/.store in=\marksslinet,
  substructure mark line type/.store in=\marksslinetype,
  substructure mark line color/.store in=\marksslinecolor,
  show superstructure shade/.store in=\showsupershade,
  superstructure shade space/.store in=\supershadespace,
  superstructure shade opacity/.store in=\supershadeopacity,
  show superstructure mark/.store in=\showmarksuper,
  superstructure mark space/.store in=\marksuperexspace,
  superstructure mark radius/.store in=\marksuperrad,
  superstructure mark line thickness/.store in=\marksuperlinet,
  superstructure mark line type/.store in=\marksuperlinetype,
  superstructure mark line color/.store in=\marksuperlinecolor,
}
\newcommand{\framestructure}[1][]{
\pgfkeys{/framestructure, default, #1}
\tikzmath{
int \storynumber, \baynumber, \columnnumber, \levelnumber, \storyminone;
int \nlevmo, \ncolmo, \iii, \j, \pileind;
real \storyheight, \baywidth, \startx, \starty, \xx, \y;
int \numberofisolators, \kiso, \isoshiftyn;
real \supportwidth, \supportheight, \isolationwidth, \isolationdepth, \isomidy;
real \foundationdepth, \massrad, \xiso, \isospace;
real \axisseperation, \linet, \baselinet, \isolinet;
real \beamlinet, \collinet;
real \rigbasestartx, \rigbaseendx, \isoboty, \isotopy, \isoshift;
real \foundboty, \foundtopy, \foundstartx, \foundendx;
int \doflocfloor, \dofloch, \dofloccolumn;
int \showaxes, \showdof, \showmass, \showsupports;
real \arrowlenratio, \minlen, \dofxx, \dofyy, \arrlen, \arrrad;
real \dofxrotation, \dofyrotation, \dofrrotation;
real \rotdofstartangle, \rotdofendangle;
int \subfloors, \deflstart, \deflstartplusone, \superstorynumber;
real \basewallstartx, \basewallstarty, \basewallendx, \basewallendy;
real \buildingwidth, \basewalldepth;
int \halfspacetype;
real \leftsoildist, \rightsoildist, \leftsoildepth, \rightsoildepth;
real \soilbelowfound, \leftcontrolx, \leftcontroly;
real \rightcontrolx, \rightcontroly;
real \rightsoilx, \rightsoily;
real \axeslenX, \axeslenY;
real \piledepth, \pilesidespace, \pilespace, \pilecoordy, \pilecoordx;
real \pilelinethickness, \pblinet;
int \showpbline;
real \latloadshift, \arrstartx, \toparrlen, \addtempy, \basearrlen;
int \latloadtype, \iarr, \showlatload;
int \substextnum;
int \showdefl;
int \istory, \ibay, \icol, \ibeam, \idefl, \jdefl;
real \drift, \deflect, \defparx, \defpary, \defparbeamx, \defparbeamy, \defbase;
real \tempdim, \fixbeamx, \fixbeamy;
int \dofcounterx, \dofcountery, \dofcounterz;
int \showengbedrock;
real \engbedrockdepth, \engbedrockleftdist, \engbedrockrightdist;
real \engbedrockstartx, \engbedrockstarty;
real \engbedrockdeltax, \engbedrockdeltay;
real \engbedrocklinewidth;
int \showtranslayer, \showssinter;
real \translayerleftstartx, \translayerleftstarty, \translayerleftdeltay;
real \translayerrightstartx, \translayerrightstarty, \translayerrightdeltay;
real \translayerlinet, \ssinterlinet;
int \showmarkss;
real \markssleftstartx, \markssleftstarty;
real \markssdeltax, \markssdeltay, \markssrad, \marksslinet, \markssexspace;
real \supershadespace;
real \supershadestartx, \supershadestarty, \supershadedeltax, \supershadedeltay;
real \supershadespace, \supershadeopacity;
int \showsupershade;
int \showlessbays, \leftbays, \rightbays, \startcol, \endcol;
int \showshearwall, \shearwallbay, \shearwallstartcolumn, \shearwallendcolumn;
real \shearwallstartx, \shearwallstarty \shearwalldeltax, \shearwalldeltay;
real \shearwalllinet;
int \showmarksuper;
real \marksuperexspace, \marksuperrad, \marksuperlinet;
real \marksuperleftstartx, \marksuperleftstarty, \marksuperdeltax, \marksuperdeltay;
%Conversion all units to the unit pt
\storyheight = \storyheight;
\baywidth = \baywidth;
\startx = \startx;
\starty = \starty;
\supportwidth = \supportwidth;
\supportheight = \supportheight;
\isolationwidth = \isolationwidth;
\isolationdepth = \isolationdepth;
\isoshift=\isoshift;
\foundationdepth = \foundationdepth;
\linet = \linet;
\beamlinet = \beamlinet;
\collinet = \collinet;
\baselinet = \baselinet;
\isolinet = \isolinet;
\massrad = \massrad;
\foundsidew = \foundsidew;
\leftsoildist = \leftsoildist;
\rightsoildist = \rightsoildist;
\leftsoildepth = \leftsoildepth;
\rightsoildepth = \rightsoildepth;
\soilbelowfound = \soilbelowfound;
\leftcontrolx = \leftcontrolx;
\leftcontroly = \leftcontroly;
\rightcontrolx = \rightcontrolx;
\rightcontroly = \rightcontroly;
\axeslenX = \axeslenX;
\axeslenY = \axeslenY;
\piledepth=\piledepth;
\pilesidespace=\pilesidespace;
\pilediameter=\pilediameter;
\pilelinethickness=\pilelinethickness;
\pblinet=\pblinet;
\latloadshift=\latloadshift;
\toparrlen=\toparrlen;
\basearrlen=\basearrlen;
\drift=\drift;
\defparx=\defparx;
\defpary=\defpary;
\defbase=\defbase;
\engbedrockdepth=\engbedrockdepth;
\engbedrockleftdist=\engbedrockleftdist;
\engbedrockrightdist=\engbedrockrightdist;
\engbedrocklinewidth=\engbedrocklinewidth;
\translayerlinet=\translayerlinet;
\ssinterlinet=\ssinterlinet;
\markssexspace=\markssexspace;
\markssrad=\markssrad;
\marksslinet=\marksslinet;
\supershadespace=\supershadespace;
\shearwalllinet=0.75pt;
\shearwalllinet=\shearwalllinet;
\marksuperexspace=\marksuperexspace;
\marksuperrad=\marksuperrad;
\marksuperlinet=\marksuperlinet;
%End conversion, All values are now in pt units.
\dofcounter=0;
\storyminone = \storynumber-1;
\columnnumber = \baynumber+1; %number of columns
\levelnumber = \storynumber+1; %number of levels
if \storynumber>1 then {\nlevmo = \levelnumber-1;} else {\nlevmo=2;};
if \baynumber>1 then {\ncolmo = \columnnumber-1;} else {\ncolmo=2;};
%% Undeflected shape coordinates
for \iii in {1,...,{\levelnumber}}{
\y{\iii} = (\iii-1)*\storyheight;
for \j in {1,...,{\columnnumber}}{
\x{\j} = (\j-1)*\baywidth;
};
};
%% Deflected shape coordinates
if equal(\showsupports,5) then
{\deflstart=\subfloors+1;
\superstorynumber=\storynumber-\subfloors;}
else
{\deflstart=1;
\superstorynumber=\storynumber;};
\deflstartplusone=\deflstart+1;
for \idefl in {1,...,{\deflstart}}{
	\deflect{\idefl} = 0;
};
for \idefl in {\deflstart+1,...,{\levelnumber}}{
	\deflect{\idefl} = (\idefl-\deflstart)*\drift;
};
\tempdim = sqrt(\defparx*\defparx + \defpary*\defpary);
\fixbeamx = \collinet/2*\defpary/\tempdim;
\fixbeamy = \collinet/2*\defparx/\tempdim;
\defparbeamx = \defparx / \storyheight * \baywidth;
\defparbeamy = \defpary / \storyheight * \baywidth;
\rigbasestartx = \x1-\supportwidth;
\rigbaseendx = \x{\columnnumber}+\supportwidth;
\isoboty = -\supportheight-\baselinet/2-\isolationdepth;
\isotopy = -\supportheight-\baselinet/2;
\foundboty = -\supportheight-\baselinet-\isolationdepth-\foundationdepth;
\foundtopy = -\supportheight-\baselinet-\isolationdepth;
\foundstartx = \x1-\foundsidew;
\foundendx = \x{\columnnumber}+\foundsidew;
\structheight=\storynumber*\storyheight;
\isomidy = \supportheight+\isolinet+\isolationdepth/2-\baselinet;
%%%Basement%%%%
\soilbelowfoundtof = \soilbelowfound+\supportheight;
\basewallstartx=-\supportwidth;
\buildingwidth=\baynumber*\baywidth;
\basewalldepth=\subfloors*\storyheight;
\basewallstarty=\basewalldepth;
\basewallendx=\buildingwidth+\supportwidth;
\basewallendy=\basewallstarty;
\rightsoilx=\buildingwidth+\rightsoildist;
\rightsoily=\basewalldepth-\rightsoildepth-\supportheight;
%Some isolator properties
\isospace = (\buildingwidth)/(\numberofisolators-1);
for \kiso in {1,...,{\numberofisolators}}{
\xiso{\kiso} = (\kiso-1)*\isospace;
};
if equal(\isoshiftyn,1) then
{\xiso{1}=\xiso{1}+\isoshift;
\xiso{\numberofisolators}=\xiso{\numberofisolators}-\isoshift;}
else {};
%%%DOFs%%%%
if greater(\doflocfloor,\storynumber) then {\doflocfloor=0;} else{};
if greater(\dofloccolumn,\columnnumber) then {\dofloccolumn=1;} else{};
\dofloch = 1+\doflocfloor;
\minlen = min(\storyheight,\baywidth);
\dofxx = \x{\dofloccolumn}+\dofoffsetratio*\minlen+\deflect{\dofloch}*\showdefl;
\dofyy = \y{\dofloch}+\dofoffsetratio*\minlen;
\arrlen = \arrowlenratio*\minlen;
\arrrad = \rotarrowlenratio*\minlen;
%\arrrad = \arrlen*0.8;
%%%%%Axes%%%%%
\Xaxesstarty = \y{1};
\Yaxesstartx = \x{1};
\Yaxesstarty = \y{\levelnumber}+\axisseperation;
if equal(\showsupports,0) then
{\Xaxesstartx = \x{\columnnumber}+\axisseperation;} else
{\Xaxesstartx = \x{\columnnumber}+\axisseperation+\supportwidth/2;};
if equal(\showsupports,1) then
{\Xaxesstartx = \x{\columnnumber}+\axisseperation+\supportwidth/2;} else
{\Xaxesstartx = \x{\columnnumber}+\axisseperation+\supportwidth;};
%%%%Piles%%%%
\pilecoordy = -\supportheight;
\pilespace = 
(2*\supportwidth+\buildingwidth-2*\pilesidespace)/(\numberofpiles-1);
for \pileind in {1,...,{\numberofpiles}}{
\pilecoordx{\pileind} = -\supportwidth+\pilesidespace+(\pileind-1)*\pilespace;
};
%%%%Lateral Loads%%%%
if or(equal(\latloadtype,1), equal(\latloadtype,3)) then {
for \iarr in {1,...,{\levelnumber}}{
	\arrstartx{\iarr}=-\toparrlen*\y{\iarr}/\structheight-\latloadshift;};
} else {};
if equal(\latloadtype,2) then {
	\arrstartx{0}=-\latloadshift;
	\y{0}=-\isomidy;
	for \iarr in {1,...,{\levelnumber}}{
		if equal(\iarr,1) then{\addtempy=\supportheight/2;} else {\addtempy=0;};
		\arrstartx{\iarr}=-\toparrlen*(\y{\iarr}+\isomidy-\addtempy)/
		(\structheight+\isomidy)-\latloadshift;
	};
} else {};
if equal(\latloadtype,4) then {
	\arrstartx{0}=-\latloadshift;
	\y{0}=-\isomidy;
	for \iarr in {1,...,{\levelnumber}}{
		if equal(\iarr,1) then{\addtempy=\supportheight/2;} else {\addtempy=0;};
		\arrstartx{\iarr}=-(\toparrlen-\basearrlen)*(\y{\iarr}+\isomidy-\addtempy)/
		(\structheight+\isomidy)-\latloadshift-\basearrlen;
	};
} else {};
%%%Engineering Bedrock Coordinates
%\engbedrocklinewidth=2pt;
%\engbedrocktext="Bedrock";
\engbedrockstartx=-\leftsoildist-\engbedrockleftdist;
\engbedrockstarty=-\soilbelowfoundtof-\engbedrockdepth;
\engbedrockdeltax=\leftsoildist+\engbedrockleftdist+
  \buildingwidth+\rightsoildist+\engbedrockrightdist;
\engbedrockdeltay=\engbedrockdepth;
\engbedrockmidx=\engbedrockstartx + \engbedrockdeltax/2;
\engbedrockmidy=\engbedrockstarty + \engbedrockdeltay/2;
%%%Transmitting Layer
%\translayerlinet=2pt;
\translayerleftstartx=-\leftsoildist-\translayerlinet/2;
\translayerleftstarty=\basewalldepth;
\translayerleftdeltay=\basewalldepth+\soilbelowfoundtof;
\translayerrightstartx=\rightsoildist+\buildingwidth+\translayerlinet/2;
\translayerrightstarty=\basewalldepth;
\translayerrightdeltay=\basewalldepth+\soilbelowfoundtof;
%%%Soil-Structure Interface
%\ssinterlinet=3pt;
\ssinterleftstartx=-\supportwidth-\baselinet/2-\ssinterlinet/2;
\ssinterrightstartx=\buildingwidth+\supportwidth+\baselinet/2+\ssinterlinet/2;
\ssinterleftstarty=\basewalldepth;
\ssinterrightstarty=\basewalldepth;
\ssinterdeltax=\ssinterrightstartx-\ssinterleftstartx;
\ssinterdeltay=\basewalldepth+\supportheight+\baselinet/2+\ssinterlinet/2;
%%%Mark for Substructure
%\markssexspace=0.3cm;
%\markssrad=0.25cm;
%\marksslinet=0.75pt;
\markssleftstartx=-\supportwidth-\markssexspace;
\markssleftstarty=\basewalldepth+\markssexspace;
\markssdeltax=\buildingwidth+2*\supportwidth+2*\markssexspace;
\markssdeltay=\basewalldepth+\supportheight+2*\markssexspace;
%%%Shade for Superstructure
%\supershadespace=5pt;
%\supershadeopacity=0.75;
\supershadestartx=-\collinet/2-\supershadespace;
\supershadestarty=\basewalldepth+\baselinet/2;
\supershadedeltax=\buildingwidth+\collinet+2*\supershadespace
 +\deflect{\levelnumber}*\showdefl;
\supershadedeltay=\superstorynumber*\storyheight+\beamlinet/2+\supershadespace;
%%% Superstructure with less bays
if equal(\showlessbays,0) then {
\leftbays=0;
\rightbays=0;};
\startcol=\leftbays+1;
\endcol=\columnnumber-\rightbays;
\superbays=\baynumber-\leftbays-\rightbays;
\superwidth=\superbays*\baywidth;
%%% Modification for the ground beam
\groundbeamshifty=\groundbeamlinet/2-\baselinet/2;
%%% Shear Wall
\showshearwall=1;
\shearwallbay=3;
\shearwallstartcolumn=\shearwallbay;
\shearwallendcolumn=\shearwallbay+1;
\shearwallstartx=\x{\shearwallstartcolumn}-\collinet/2;
\shearwallstarty=\y{1}+\baselinet/2-\shearwalllinet/2;
\shearwalldeltax=\baywidth+\collinet;
if equal(\superstorynumber,0) then {
\shearwalldeltay=\y{\levelnumber};} else {\shearwalldeltay=\y{\levelnumber}-\baselinet/2+\beamlinet/2;};
%%%Mark for Superstructure
%\showmarksuper=1;
%\marksuperexspace=0.3cm;
%\marksuperrad=0.25cm;
%\marksuperlinet=0.75pt;
\marksuperleftstartx=\x{\startcol}-\marksuperexspace;
\marksuperleftstarty=\basewalldepth+\marksuperexspace/2;
\marksuperdeltax=\superwidth+2*\marksuperexspace;
\marksuperdeltay=\y{\levelnumber}-\y{\deflstart}+\marksuperexspace/2;
}   %End tikzmath

\begin{scope}[x=1pt, y=1pt, xshift=\startx, yshift=\starty, rotate=0];
	% Drawing everything in pt units


%%%% Lateral Laods
\ifthenelse{\showlatload=1}{
\ifthenelse{\latloadtype=1}{
	\foreach \iarr in {2,...,{\levelnumber}}{
		\tikzmath{\substextnum=\iarr-1;}
		\draw[arrows={-[length=5pt, width=4pt]>}, line width=1pt]
		(\arrstartx{\iarr},\y{\iarr}) --
		node[anchor=south, pos=0.4, inner sep=1pt]{$F_{\substextnum}$}
		(-\latloadshift,\y{\iarr});}
	\draw (\arrstartx{1},\y{1}) -- (\arrstartx{\levelnumber},\y{\levelnumber});
	\draw (\arrstartx{1},\y{1}) -- (\arrstartx{1},\y{\levelnumber});
}{;}

\ifthenelse{\latloadtype=2}{
	\foreach \iarr in {1,...,{\levelnumber}}{
		\tikzmath{\substextnum=\iarr-1;}
		\ifthenelse{\iarr=1}{
			\draw[arrows={-[length=2pt, width=2pt]>}, line width=0.5pt]
			(\arrstartx{\iarr},\y{\iarr}-\supportheight/2) --
			node[anchor=east, pos=0, inner sep=1pt]{$F_{\text{b}}$}
			(-\latloadshift,\y{\iarr}-\supportheight/2);
		}{
			\draw[arrows={-[length=5pt, width=4pt]>}, line width=1pt]
			(\arrstartx{\iarr},\y{\iarr}) --
			node[anchor=south, pos=0.4, inner sep=1pt]{$F_{\substextnum}$}
			(-\latloadshift,\y{\iarr});
		}
	}
	\draw (\arrstartx{0},\y{0}) -- (\arrstartx{\levelnumber},\y{\levelnumber});
	\draw (\arrstartx{0},\y{0}) -- (\arrstartx{0},\y{\levelnumber});
}{;}

\ifthenelse{\latloadtype=4}{
	\foreach \iarr in {1,...,{\levelnumber}}{
		\tikzmath{\substextnum=\iarr-1;}
		\ifthenelse{\iarr=1}{
			\draw[arrows={-[length=5pt, width=4pt]>}, line width=1pt]
			(\arrstartx{\iarr},\y{\iarr}-\supportheight/2) --
			node[anchor=south, pos=0.4, inner sep=1pt]{$F_{\text{b}}$}
			(-\latloadshift,\y{\iarr}-\supportheight/2);
		}{
			\draw[arrows={-[length=5pt, width=4pt]>}, line width=1pt]
			(\arrstartx{\iarr},\y{\iarr}) --
			node[anchor=south, pos=0.4, inner sep=1pt]{$F_{\substextnum}$}
			(-\latloadshift,\y{\iarr});
		}
	}
	\draw (\arrstartx{0},\isoboty) -- (\arrstartx{1},\y{1}-\supportheight/2)
	-- (\arrstartx{\levelnumber},\y{\levelnumber});
	\draw (\arrstartx{0},\isoboty) -- (\arrstartx{0},\y{\levelnumber});
}{;}


\ifthenelse{\latloadtype=3}{
	\foreach \iarr in {2,...,{\levelnumber}}{
		\tikzmath{\substextnum=\iarr-1;}
		\draw[arrows={-[length=5pt, width=4pt]>}, line width=1pt]
		(\arrstartx{\iarr},\y{\iarr}) --
		node[anchor=south, pos=0.4, inner sep=1pt]{$F_{\substextnum}$}
		(-\latloadshift,\y{\iarr});}
	\draw (\arrstartx{1},\y{1}) -- (\arrstartx{\levelnumber},\y{\levelnumber});
	\draw (\arrstartx{1},\y{1}) -- (\arrstartx{1},\y{\levelnumber});
	\draw[arrows={-[length=5pt, width=4pt]>}, line width=1pt]
	(-\latloadshift-1cm,-\supportheight/2)--
	node[anchor=south, pos=0.4, inner sep=1pt]{$F_{\text{b}}$}
	++(1cm,0);
}{;}

\ifthenelse{\latloadtype=5}
{
	\foreach \istory in {2,...,{\levelnumber}}
	{
		\tikzmath{\substextnum=\istory-1;}
		\draw [arrows={-[length=5pt, width=4pt]>}, line width=1pt]
		(\x{1}+\deflect{\istory}-\toparrlen,\y{\istory}) --
			node[anchor=south, pos=0.4, inner sep=1pt]{$F_{\substextnum}$}		
		(\x{1}+\deflect{\istory}-0.1cm, \y{\istory});
	}
}{;}
}{;}


%%%% Different Support Types
%% 0. No Supports
\ifthenelse{\showsupports=0}{
}{}

%% 1. Individual fixed supports
\ifthenelse{\showsupports=1}{
	\foreach \j in {1,...,{\columnnumber}}
	{
		\fill [gray] (\x{\j},\y{1}) +({-\supportwidth/2},-\supportheight)
		rectangle +({\supportwidth/2},\y{1});
		\draw [line width = \baselinet]
		(\x{\j},\y{1}) +({-\supportwidth/2},\y{1}) -- +({\supportwidth/2},\y{1});
	}
}{}

%% 2. Support as continuous plate
\ifthenelse{\showsupports=2}{
	\fill [gray] (\rigbasestartx, 0) rectangle (\rigbaseendx,-\supportheight);
	\draw [line width = \baselinet] (\rigbasestartx, 0) -- (\rigbaseendx,0);
}{}


%% 3. Support as ...
\ifthenelse{\showsupports=3}{
	\foreach \j in {1,...,{\columnnumber}}
	{\fill [fill=gray, draw=black, line width = \baselinet] (\rigbasestartx, 0) 
		rectangle
		(\rigbaseendx,-\supportheight);}
}{}


%%% 4. Base Isolation
\ifthenelse{\showsupports=4}{
	\fill [fill=gray, draw=black, line width = \baselinet] (\rigbasestartx, 0) 
	rectangle
	(\rigbaseendx,-\supportheight);
	
	\foreach \j in {1,...,{\numberofisolators}}
	{\filldraw [fill = white!20!black, draw=black, line width=\isolinet]
		(\xiso{\j},\y{1}) +(-\isolationwidth/2,\isoboty) rectangle
		+(\isolationwidth/2,\isotopy);}
	
	\fill [fill=gray]
	(\foundstartx, \foundboty) rectangle (\foundendx,\foundtopy);
	\draw [line width = \baselinet] (\foundstartx, \foundtopy) -- (\foundendx, 
	\foundtopy);
}{}

%%5. Pile Supports
\ifthenelse{\showsupports=5}{
	
	%%% Rectangular half space
	\ifthenelse{\halfspacetype=1}{
	\path[fill=brown!75!white]
	{(-\leftsoildist,\basewalldepth) -- ++(0,-\basewalldepth) -- ++(0,-\soilbelowfoundtof)
		-- ++(\buildingwidth,0) -- ++(\leftsoildist,0) -- ++(\rightsoildist,0)
		-- ++(0,\soilbelowfoundtof)-- ++(0,\basewalldepth) }
	-- cycle;}{}
	
	\ifthenelse{\showtranslayer=1}{
	%%% Transmitting Layers
	\draw[line width=\translayerlinet, \translayercolor, \translayerlinetype]
	(\translayerleftstartx,\translayerleftstarty) -- ++(0,-\translayerleftdeltay);
	
	\draw[line width=\translayerlinet, \translayercolor, \translayerlinetype]
	(\translayerrightstartx,\translayerrightstarty) -- ++(0,-\translayerrightdeltay);}{}
	
	%%% Random half space
	\ifthenelse{\halfspacetype=2}{
	\path[fill=brown!75!white] decorate[decoration={random steps, segment length=0.25cm}]
		{(-\leftsoildist,\basewalldepth) -- ++(0,-\leftsoildepth)
		..controls++(0,-\leftcontroly) and ++(-\leftcontrolx,0)..(0,-\soilbelowfoundtof)
		-- ++(\buildingwidth,0)
		..controls ++(\rightcontrolx,0) and ++(0,-\rightcontroly) .. (\rightsoilx,\rightsoily)
		-- ++(0,\supportheight)-- ++(0,\rightsoildepth)}
	-- cycle;}{}
	
	
	%%% Top thin line
	\draw[line width=0.5pt, color=black]
	(0,\basewalldepth) -- ++(-\leftsoildist,0);
	
	\draw[line width=0.5pt, color=black]
	(\buildingwidth,\basewalldepth) -- ++(\rightsoildist,0);
	
	%% White rectangle
	\path[fill=white](0,\basewalldepth+1pt) -- ++(\buildingwidth,0)
	--++(0,-\basewalldepth-1pt) -- ++(-\buildingwidth,0) -- cycle;
	
	
	%%% Piles
	\ifthenelse{\showpiles=1}{
		\foreach \pileind in {1,...,{\numberofpiles}}
		{\pilesupport[x coordinate=\pilecoordx{\pileind}, y coordinate=\pilecoordy,
			pile depth=\piledepth, pile diameter=\pilediameter,
			pile line thickness=\pilelinethickness, fill color=gray,
			draw boundary line=\showpbline,
			boundary line thickness=\pblinet,
			boundary line color=\pblinecolor,
			boundary line type=\pblinetype,]}
	}{}
	
	
	%%% Retaining walls and mat foundations
	\filldraw[line width=\baselinet, fill=gray!65]
	(\basewallstartx,\basewallstarty) -- ++(\supportwidth,0)
	-- ++(0,-\basewalldepth) -- ++(\buildingwidth,0)
	-- ++(0,\basewalldepth) -- ++(\supportwidth,0)
	-- ++(0,-\basewalldepth-\supportheight) -- (\basewallstartx,-\supportheight)
	-- cycle;
	
	\ifthenelse{\showssinter=1}{
	%%%Soil-Structure Interface
	\draw[line width=\ssinterlinet, \ssintercolor, \ssinterlinetype]
	(\ssinterleftstartx,\ssinterleftstarty) -- ++(0,-\ssinterdeltay) 
	-- ++ (\ssinterdeltax,0) -- ++(0,\ssinterdeltay);}{}

	
	%% Engineering Bedrock
	\ifthenelse{\showengbedrock=1}{
	\fill[gray]
		(\engbedrockstartx, \engbedrockstarty) rectangle
		++(\engbedrockdeltax, \engbedrockdeltay);
		
	\draw[line width=\engbedrocklinewidth]
		(\engbedrockstartx,\engbedrockstarty) ++(0, \engbedrockdeltay)
		-- ++(\engbedrockdeltax,0);
		
	\node[align=center, scale=1] at
		(\engbedrockmidx, \engbedrockmidy) {\engbedrocktext};}{}
	
}{}


%\draw (0,0)-- (0.5cm,2cm);
%%% Draw Frame - Undeflected shape
\ifthenelse{\showdefl=0}{

%%Substructure
\ifthenelse{\subfloors>0}{
	
	%Draw the beam at the ground level
	\draw [line width = \groundbeamlinet, \beamcolor]
	(\x{1}+\baselinet/2, \y{\deflstart}-\groundbeamshifty)
	-- (\x{\columnnumber}-\baselinet/2, \y{\deflstart}-\groundbeamshifty);
	
	%Draw the beams except the ground level beam
	\ifthenelse{\subfloors>1}{
		\foreach \iii in {2,...,{\subfloors}}
		{\draw [blue, line width = \beamlinet, \beamcolor]
			(\x{1}+\baselinet/2,\y{\iii})
			-- (\x{\columnnumber}-\baselinet/2,\y{\iii});}
	}{}
	
	%Draw first column - No Need to Draw when bays are reduced
	\ifthenelse{\leftbays=0}{
	\ifthenelse{\superstorynumber>0}{
	\draw [line width = \collinet, \columncolor]
	(\x{1},\y{1}-\baselinet/2) -- (\x{1}, \y{\deflstart}+\beamlinet/2);}{}}{}
	
	%Draw the columns between the first and the last column
	\foreach \j in {2,...,{\ncolmo}}
	{\draw [line width = \collinet, \columncolor]
		(\x{\j},\y{1}+\baselinet/2)
		-- (\x{\j},\y{\deflstart}+\baselinet/2);}
	
	%Draw the last column - No Need to Draw when bays are reduced
	\ifthenelse{\rightbays=0}{
	\ifthenelse{\superstorynumber>0}{
	\draw [line width = \collinet, \columncolor]
	(\x{\columnnumber}, \y{1}-\baselinet/2)
	-- (\x{\columnnumber}, \y{\deflstart}+\beamlinet/2);}{}}{}	
	
}{}


%%% Superstructure
\ifthenelse{\superstorynumber>0}{
%% Draw beams up the top level minus one
\ifthenelse{\superstorynumber>1}{
\foreach \iii in {\deflstartplusone,...,{\nlevmo}}
{\draw [\beamcolor, line width = \beamlinet]
(\x{\startcol},\y{\iii}) -- (\x{\endcol},\y{\iii});}
}{}

%% Draw the beam the the top floor
\draw [line width = \beamlinet, \beamcolor]
(\x{\startcol}-\collinet/2, \y{\levelnumber})
-- (\x{\endcol}+\collinet/2, \y{\levelnumber});

%% Draw the first column
\draw [line width = \collinet, \columncolor]
(\x{\startcol},\y{\deflstart}-\baselinet/2)
-- (\x{\startcol}, \y{\levelnumber}+\beamlinet/2);

%% Columns other than first and last columns
\foreach \j in {\startcol,...,{\endcol}}
{\draw [line width = \collinet, \columncolor]
	(\x{\j},\y{\deflstart}-\baselinet/2)
	-- (\x{\j},\y{\levelnumber}+\beamlinet/2);}
}{}

%%%% Shear Wall
\ifthenelse{\showshearwall=1}{
\filldraw[fill=gray, line width=\shearwalllinet]
(\shearwallstartx, \shearwallstarty)
rectangle ++(\shearwalldeltax,\shearwalldeltay);}{}

%%% Mark for Substructure
\ifthenelse{\superstorynumber>0}{
\ifthenelse{\showmarksuper=1}{
	\draw[rounded corners=\marksuperrad,
	line width=\marksuperlinet, color=\marksuperlinecolor, \marksuperlinetype]
	(\marksuperleftstartx,\marksuperleftstarty)
	rectangle ++(\marksuperdeltax,\marksuperdeltay);}{}}{}

%% Masses for Undeflected Shape
\ifthenelse{\showmass=1}{
\ifthenelse{\showsupports=5}{
	
	%% Masses for superstructure
	\ifthenelse{\superstorynumber>0}{
	\foreach \iii in {\useeval{2+\subfloors},...,{\levelnumber}}
	{\foreach \j in {\startcol,...,{\endcol}}{
	\ifthenelse{\showshearwall=1}{
		
		\ifthenelse{\j=\shearwallstartcolumn}{
		\shade[ball color=\masscolorsuper]
		(\x{\j}-\collinet/2,\y{\iii}) circle (\massrad);}{
		
		\ifthenelse{\j=\shearwallendcolumn}{
		\shade[ball color=\masscolorsuper]
		(\x{\j}+\collinet/2,\y{\iii}) circle (\massrad);}{
		
		\shade[ball color=\masscolorsuper]
		(\x{\j},\y{\iii}) circle (\massrad);}};}{}
	}}}{}

	%% Masses for substructure
	\ifthenelse{\subfloors>0}{
	\foreach \iii in {2,...,\useeval{\deflstart}}
	{\foreach \j in {2,...,\useeval{\columnnumber-1}}{
	%% Masses at ground level
	\ifthenelse{\iii=\deflstart}{
		
		\ifthenelse{\showshearwall=1}{
	
		\ifthenelse{\j=\shearwallstartcolumn}{
		\shade[ball color=\masscolorsub]
		(\x{\j}-\collinet/2,\y{\iii}-\groundbeamshifty) circle (\massrad);}{
		
		\ifthenelse{\j=\shearwallendcolumn}{
		\shade[ball color=\masscolorsub]
		(\x{\j}+\collinet/2,\y{\iii}-\groundbeamshifty) circle (\massrad);}{
	
		\shade[ball color=\masscolorsub]
		(\x{\j},\y{\iii}-\groundbeamshifty) circle (\massrad);}};}{}
	}
	%%Masses at sub levels other than grond level
	{
		\ifthenelse{\showshearwall=1}{
		
		\ifthenelse{\j=\shearwallstartcolumn}{
		\shade[ball color=\masscolorsub]
		(\x{\j}-\collinet/2,\y{\iii}) circle (\massrad);}{
		
		\ifthenelse{\j=\shearwallendcolumn}{
		\shade[ball color=\masscolorsub]
		(\x{\j}+\collinet/2,\y{\iii}) circle (\massrad);}{
			
		\shade[ball color=\masscolorsub]
		(\x{\j},\y{\iii}) circle (\massrad);}};}{}
	}
	}
	
	}}{}
}
{
	\foreach \iii in {2,...,{\levelnumber}}
	{\foreach \j in {1,...,{\columnnumber}}{
		\shade[ball color=\masscolorsuper]
		(\x{\j},\y{\iii}) circle (\massrad);
	}}
}
}{}


}{}

%%% Draw Frame - Undeflected shape but beams and columns are drawn seperately
\ifthenelse{\showdefl=3}{
	\foreach \istory in {1,...,{\storynumber}}{
		\foreach \icol in {1,...,{\columnnumber}}{
			\draw [line width = \collinet, \columncolor]
			(\x{\icol},\y{\istory}) -- (\usevar\x{\icol}, \usevar\y{\istory+1});}
		\foreach \ibeam in {1,...,{\baynumber}}		{
			\draw [line width = \beamlinet, \beamcolor]
			(\x{\ibeam}-\collinet/2,\usevar\y{\istory+1}) --
			(\usevar\x{\ibeam+1}+\collinet/2, \usevar\y{\istory+1});}}
}{}




%\node at (10,10) {\deflstart};
%\node at (50,50) {\deflstartplusone};
%\node at (100,100) {\superstorynumber};
%%% Draw Frame - Deflected shape
\ifthenelse{\showdefl=1}{
	
%%% Substructure - If the support type is 5
\ifthenelse{\showsupports=5}{
\ifthenelse{\subfloors>1}{
	
%Draw the beams except the ground level beam
\foreach \iii in {2,...,{\subfloors}}
{\draw [blue, line width = \beamlinet, \beamcolor]
(\x{1},\y{\iii}) -- (\x{\columnnumber},\y{\iii});}

%Draw the beam at the ground level
\draw [line width = \beamlinet, \beamcolor]
(\x{1}-\collinet/2, \y{\deflstart})
-- (\x{\columnnumber}+\collinet/2, \y{\deflstart});
	
%Draw first column
\draw [line width = \collinet, \columncolor]
(\x{1},\y{1}-\baselinet/2) -- (\x{1}, \y{\deflstart}+\beamlinet/2);

%Draw the columns between the first and the last column
\foreach \j in {2,...,{\ncolmo}}
{\draw [line width = \collinet, \columncolor]
(\x{\j},\y{1}-\baselinet/2) -- (\x{\j},\y{\deflstart}+\beamlinet/2);}
	
%Draw the last column
\draw [line width = \collinet, \columncolor]
(\x{\columnnumber}, \y{1}-\baselinet/2)
-- (\x{\columnnumber}, \y{\deflstart}+\beamlinet/2);
	
}{}
}{}



	
	
%%%Superstructure

%% If the superstory number is different then zero, first draw the first story
\ifthenelse{\superstorynumber>0}{
%%Beams at the second level only
\foreach \ibeam in {1,...,{\baynumber}}
{
	\draw [red, line width = \beamlinet, \beamcolor]
	(\x{\ibeam}-\fixbeamx+\deflect{\deflstartplusone}, \y{\deflstartplusone}+\fixbeamy)
	.. controls +(\defparbeamy,-\defparbeamx) and +(-\defparbeamy, \defparbeamx) ..
	(\usevar\x{\ibeam+1}+\fixbeamx + \deflect{\deflstartplusone}, \y{\deflstartplusone}-\fixbeamy);
}
%% Columns at the first story only
\foreach \icol in {1,...,{\columnnumber}}
{
	\draw [line width = \collinet, \columncolor]
	(\x{\icol}+\deflect{\deflstart},\y{\deflstart})
	.. controls +(0.0cm, \defbase) and +(-\defparx, -\defpary) ..
	(\x{\icol}+\deflect{\deflstartplusone}, \y{\deflstartplusone}+\beamlinet/2);
}}{}

%% If story number is larger than two, draw the storys other than first and last story
\ifthenelse{\superstorynumber>1}{
	\ifthenelse{\superstorynumber>2}{
		\foreach \istory in {\deflstartplusone,...,{\storyminone}}
		{
			\foreach \ibeam in {1,...,{\baynumber}}
			{
				\draw [blue, line width = \beamlinet, \beamcolor]
				(\x{\ibeam}-\fixbeamx+\usevar\deflect{\istory+1}, \usevar\y{\istory+1}+\fixbeamy)
				.. controls +(\defparbeamy,-\defparbeamx) and +(-\defparbeamy, +\defparbeamx)
				.. (\usevar\x{\ibeam+1}+\fixbeamx+\usevar\deflect{\istory+1},
				\usevar\y{\istory+1}-\fixbeamy);
			}
			\foreach \icol in {1,...,{\columnnumber}}
			{
				\draw [line width = \collinet, \columncolor]
				(\x{\icol}+\deflect{\istory},\y{\istory})
				.. controls +(\defparx, \defpary) and +(-\defparx, -\defpary) ..
				(\usevar\x{\icol}+\usevar\deflect{\istory+1}, \usevar\y{\istory+1});
			}
		}
	}{}
	
	%% Draw the last story
	\foreach \istory in {\storynumber}
	{
		\foreach \ibeam in {1,...,{\baynumber}}
		{
			\draw [line width = \beamlinet, \beamcolor]
			(\x{\ibeam}-\fixbeamx+\usevar\deflect{\istory+1}, \usevar\y{\istory+1}+\fixbeamy)
			.. controls +(\defparbeamy,-\defparbeamx) and +(-\defparbeamy, +\defparbeamx) ..
			(\usevar\x{\ibeam+1}+\fixbeamx+\usevar\deflect{\istory+1},\usevar\y{\istory+1}-\fixbeamy);
		}
		\foreach \icol in {1,...,{\columnnumber}}
		{
			\draw [line width = \collinet, \columncolor]
			(\x{\icol}+\deflect{\istory},\y{\istory})
			.. controls +(\defparx, \defpary) and +(-\defparx, -\defpary) ..
			(\usevar\x{\icol}+\usevar\deflect{\istory+1}, +\usevar\y{\istory+1}+\beamlinet/2);
		}
	}	
	
}{}

%% Masses for deflected shape
\ifthenelse{\showmass=1}{
	\ifthenelse{\showsupports=5}{
		\foreach \iii in {\useeval{2+\subfloors},...,{\levelnumber}}
		{\foreach \j in {1,...,{\columnnumber}}{
			\shade[ball color=black] (\x{\j} + \deflect{\iii}, \y{\iii}) circle (\massrad);
		}}
		\foreach \iii in {2,...,\useeval{\subfloors+1}}
			{\foreach \j in {2,...,\useeval{\columnnumber-1}}{
				\shade[ball color=black] (\x{\j} + \deflect{\iii}, \y{\iii}) circle (\massrad);
			}}
	}
	{
	\foreach \iii in {2,...,{\levelnumber}}
		{\foreach \j in {1,...,{\columnnumber}}{
			\shade[ball color=black] (\x{\j} + \deflect{\iii}, \y{\iii}) circle (\massrad);
		}}
	}
}{}
	
	
}{}

\ifthenelse{\showsupports=5}{
	%White Shading for Superstructure
	\ifthenelse{\showsupershade=1}{
		\fill[fill=white, opacity=\supershadeopacity]
		(\supershadestartx,\supershadestarty)
		rectangle ++(\supershadedeltax,\supershadedeltay);}{}
	
	%%% Mark for Substructure
	\ifthenelse{\showmarkss=1}{
		\draw[rounded corners=\markssrad,
		line width=\marksslinet, color=\marksslinecolor, \marksslinetype]
		(\markssleftstartx,\markssleftstarty)
		rectangle ++(\markssdeltax,-\markssdeltay);}{}
}{}

%%% Global Axes
\ifthenelse{\showaxes=1}{
\tikzset{>={Classical TikZ Rightarrow[scale=1, width=3, length=5, bend]}};
\draw [{->}] (\Xaxesstartx,\Xaxesstarty) -- +(\axeslenX,0) node[above]{$X$};
\draw [{->}] (\Yaxesstartx,\Yaxesstarty) -- +(0,\axeslenY) node[right]{$Y$};
}{}




%%% Degrees of Freedom

%% DOF at only one joint
\ifthenelse{\showdof=1}{
\dofs[startx = \dofxx, starty = \dofyy,
  length x = \arrlen,  length y = \arrlen,
  offset x=\dofoffsetx, offset y=\dofoffsety,
  offset rotation xdir=\dofrotoffsetx, offset rotation ydir=\dofrotoffsety,
  start angle = \rotdofstartangle, end angle = \rotdofendangle, radius = \arrrad,
  xstring = \dofxstr, ystring = \dofystr, rstring = \dofrstr,
  font size = \normalsize, font scale = \doftextratio,
  rotation=0, font rotation=0,
  font rotation for x = \dofxrotation,
  font rotation for y = \dofyrotation,
  font rotation for r = \dofrrotation,
  dof arrow size ratio=\dofarrowratio,
  show dofx = \shodofx, show dofy = \shodofy, show dofr = \shodofr,
  dofx position=\dofposx, dofy position=\dofposy, dofr position=\dofposr,
  dofx inner sep=\dofinnersepx, dofy inner sep=\dofinnersepx,
  dofr inner sep=\dofinnersepr,]  
}{}

\ifthenelse{\showdof=2}{

\ifthenelse{\showsupports=5}{
\foreach \iii in {\useeval{2+\subfloors},...,{\levelnumber}}
{\foreach \j in {1,...,{\columnnumber}}{

\ifthenelse{\shodofx=1}{
	\ifthenelse{\shodofy=1}{
		\ifthenelse{\shodofr=1}{
			\tikzmath{
				\dofcounterx=(\iii-2)*\columnnumber*3+ (\j-1)*3 + 1;
				\dofcountery=\dofcounterx+1);
				\dofcounterz=\dofcountery+1;}
		}{}	
	}{}	
}{}

\ifthenelse{\shodofx=1}{
	\ifthenelse{\shodofy=1}{
		\ifthenelse{\shodofr=0}{
			\tikzmath{
				\dofcounterx=(\iii-2)*\columnnumber*2+ (\j-1)*2 + 1;
				\dofcountery=\dofcounterx+1);}
		}{}	
	}{}	
}{}

\ifthenelse{\shodofx=1}{
	\ifthenelse{\shodofy=0}{
		\ifthenelse{\shodofr=1}{
			\tikzmath{
				\dofcounterx=(\iii-2)*\columnnumber*2+ (\j-1)*2 + 1;
				\dofcounterz=\dofcounterx+1);}
		}{}	
	}{}	
}{}

\ifthenelse{\shodofx=0}{
	\ifthenelse{\shodofy=1}{
		\ifthenelse{\shodofr=1}{
			\tikzmath{
				\dofcountery=(\iii-2)*\columnnumber*2+ (\j-1)*2 + 1;
				\dofcounterz=\dofcountery+1);}
		}{}	
	}{}	
}{}

\ifthenelse{\shodofx=0}{
	\ifthenelse{\shodofy=0}{
		\ifthenelse{\shodofr=1}{
			\tikzmath{
				\dofcounterz=(\iii-2)*\columnnumber*1+ (\j-1)*1 + 1;}
		}{}	
	}{}	
}{}


\dofs[startx = \x{\j}+\deflect{\iii}, starty = \y{\iii},
  length x = \arrlen,  length y = \arrlen,
  offset x=\dofoffsetx, offset y=\dofoffsety,
  offset rotation xdir=\dofrotoffsetx, offset rotation ydir=\dofrotoffsety,
  start angle = \rotdofstartangle, end angle = \rotdofendangle, radius = \arrrad,
  xstring = \dofcounterx, ystring = \dofcountery, rstring = \dofcounterz,
  font size = \normalsize, font scale = \doftextratio,
  rotation=0, font rotation=0,
  font rotation for x = \dofxrotation,
  font rotation for y = \dofyrotation,
  font rotation for r = \dofrrotation,
  dof arrow size ratio=\dofarrowratio,
  show dofx = \shodofx, show dofy = \shodofy, show dofr = \shodofr,
  dofx position=\dofposx, dofy position=\dofposy, dofr position=\dofposr,
  dofx inner sep=\dofinnersepx, dofy inner sep=\dofinnersepx,
  dofr inner sep=\dofinnersepr,]  

}}

\foreach \iii in {2,...,\useeval{\subfloors+1}}
{\foreach \j in {2,...,\useeval{\columnnumber-1}}{
		
\ifthenelse{\shodofx=1}{
	\ifthenelse{\shodofy=1}{
		\ifthenelse{\shodofr=1}{
			\tikzmath{
				\dofcounterx=(\iii-2)*\columnnumber*3+ (\j-1)*3 + 1;
				\dofcountery=\dofcounterx+1);
				\dofcounterz=\dofcountery+1;}
		}{}	
	}{}	
}{}

\ifthenelse{\shodofx=1}{
	\ifthenelse{\shodofy=1}{
		\ifthenelse{\shodofr=0}{
			\tikzmath{
				\dofcounterx=(\iii-2)*\columnnumber*2+ (\j-1)*2 + 1;
				\dofcountery=\dofcounterx+1);}
		}{}	
	}{}	
}{}

\ifthenelse{\shodofx=1}{
	\ifthenelse{\shodofy=0}{
		\ifthenelse{\shodofr=1}{
			\tikzmath{
				\dofcounterx=(\iii-2)*\columnnumber*2+ (\j-1)*2 + 1;
				\dofcounterz=\dofcounterx+1);}
		}{}	
	}{}	
}{}

\ifthenelse{\shodofx=0}{
	\ifthenelse{\shodofy=1}{
		\ifthenelse{\shodofr=1}{
			\tikzmath{
				\dofcountery=(\iii-2)*\columnnumber*2+ (\j-1)*2 + 1;
				\dofcounterz=\dofcountery+1);}
		}{}	
	}{}	
}{}

\ifthenelse{\shodofx=0}{
	\ifthenelse{\shodofy=0}{
		\ifthenelse{\shodofr=1}{
			\tikzmath{
				\dofcounterz=(\iii-2)*\columnnumber*1+ (\j-1)*1 + 1;}
		}{}	
	}{}	
}{}


\dofs[startx = \x{\j}+\deflect{\iii}, starty = \y{\iii},
  length x = \arrlen,  length y = \arrlen,
  offset x=\dofoffsetx, offset y=\dofoffsety,
  offset rotation xdir=\dofrotoffsetx, offset rotation ydir=\dofrotoffsety,
  start angle = \rotdofstartangle, end angle = \rotdofendangle, radius = \arrrad,
  xstring = \dofcounterx, ystring = \dofcountery, rstring = \dofcounterz,
  font size = \normalsize, font scale = \doftextratio,
  rotation=0, font rotation=0,
  font rotation for x = \dofxrotation,
  font rotation for y = \dofyrotation,
  font rotation for r = \dofrrotation,
  dof arrow size ratio=\dofarrowratio,
  show dofx = \shodofx, show dofy = \shodofy, show dofr = \shodofr,
  dofx position=\dofposx, dofy position=\dofposy, dofr position=\dofposr,
  dofx inner sep=\dofinnersepx, dofy inner sep=\dofinnersepx,
  dofr inner sep=\dofinnersepr,] 	
	
}}
}


{
\foreach \iii in {2,...,{\levelnumber}}
{\foreach \j in {1,...,{\columnnumber}}{


\ifthenelse{\shodofx=1}{
	\ifthenelse{\shodofy=1}{
		\ifthenelse{\shodofr=1}{
			\tikzmath{
				\dofcounterx=(\iii-2)*\columnnumber*3+ (\j-1)*3 + 1;
				\dofcountery=\dofcounterx+1);
				\dofcounterz=\dofcountery+1;}
		}{}	
	}{}	
}{}

\ifthenelse{\shodofx=1}{
	\ifthenelse{\shodofy=1}{
		\ifthenelse{\shodofr=0}{
			\tikzmath{
				\dofcounterx=(\iii-2)*\columnnumber*2+ (\j-1)*2 + 1;
				\dofcountery=\dofcounterx+1);}
		}{}	
	}{}	
}{}

\ifthenelse{\shodofx=1}{
	\ifthenelse{\shodofy=0}{
		\ifthenelse{\shodofr=1}{
			\tikzmath{
				\dofcounterx=(\iii-2)*\columnnumber*2+ (\j-1)*2 + 1;
				\dofcounterz=\dofcounterx+1);}
		}{}	
	}{}	
}{}

\ifthenelse{\shodofx=0}{
	\ifthenelse{\shodofy=1}{
		\ifthenelse{\shodofr=1}{
			\tikzmath{
				\dofcountery=(\iii-2)*\columnnumber*2+ (\j-1)*2 + 1;
				\dofcounterz=\dofcountery+1);}
		}{}	
	}{}	
}{}

\ifthenelse{\shodofx=0}{
	\ifthenelse{\shodofy=0}{
		\ifthenelse{\shodofr=1}{
			\tikzmath{
				\dofcounterz=(\iii-2)*\columnnumber*1+ (\j-1)*1 + 1;}
		}{}	
	}{}	
}{}


%\node at (\x{\j}+\deflect{\iii},\y{\iii}) {aa};

\dofs[startx = \x{\j}+\deflect{\iii}, starty = \y{\iii},
length x = \arrlen,  length y = \arrlen,
offset x=\dofoffsetx, offset y=\dofoffsety,
offset rotation xdir=\dofrotoffsetx, offset rotation ydir=\dofrotoffsety,
start angle = \rotdofstartangle, end angle = \rotdofendangle, radius = \arrrad,
xstring = \dofcounterx, ystring = \dofcountery, rstring = \dofcounterz,
font size = \normalsize, font scale = \doftextratio,
rotation=0, font rotation=0,
font rotation for x = \dofxrotation,
font rotation for y = \dofyrotation,
font rotation for r = \dofrrotation,
dof arrow size ratio=\dofarrowratio,
show dofx = \shodofx, show dofy = \shodofy, show dofr = \shodofr,
dofx position=\dofposx, dofy position=\dofposy, dofr position=\dofposr,
dofx inner sep=\dofinnersepx, dofy inner sep=\dofinnersepx,
dofr inner sep=\dofinnersepr,] 	
	
}}
}
}{}



\end{scope}
}