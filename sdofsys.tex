\pgfkeys{
 /sdofsys/.is family, /sdofsys,
  default/.style={
  startx = 0cm, starty = 0cm,
  support width = 4.0cm,
  support height = 2cm,
  vertical support line thickness = 1pt,
  horizontal support line thickness = 1pt,
  vertical support shade thickness = 0.3cm,
  horizontal support shade thickness = 0.3cm,    
  mass line thickness = 1pt,
  mass height = 1.5cm,
  mass width = 2cm,
  massx = 1.5cm,
  mass text = $m$,
  spring and damper line thickness = 0.5pt,
  spring location ratio = 0.25,
  spring ratio = 0.15,
  spring zigzag amplitude = 0.1cm,
  spring zigzag segment length = 0.2cm,
  spring text = $k$,
  spring text side = above,
  damper width ratio = 0.15,
  damper height ratio = 0.15,
  damper piston size ratio = 0.7,
  damper piston position ratio = 0.5,
  damper text = $c$,
  damper text side = 1,
  wheel line thickness = 1pt,
  wheel radius = 0.2cm,
  wheel location ratio = 0.25,
  wheel size ratio = 1,
  disp line location ratio = 1,
  disp line = 0.8cm,
  disp arrow = 1cm,
  disp text = $x(t)$,
  disp line thickness = 1pt,
  forcex location ratio = 1,
  forcey location ratio = 0.5,  
  force length = 1cm,
  force line thickness = 1pt,
  force text = $f(t)$,
  display displacement = 1,
  display force = 1},
  startx/.store in = \startcoordx, starty/.store in = \startcoordy,	
  support width/.store in = \suppwidth,	
  support height/.store in = \suppheight,	
  vertical support line thickness/.store in = \vertsupplinethk,	
  horizontal support line thickness/.store in = \horsupplinethk,	
  vertical support shade thickness/.store in = \vertsuppshadethk,	
  horizontal support shade thickness/.store in = \horsuppshadethk,	
  mass line thickness/.store in = \masslinethk,	
  mass height/.store in = \massheight,	
  mass width/.store in = \masswidth,	
  massx/.store in = \masscoordx,  	
  mass text/.store in = \masstext,	
  spring and damper line thickness/.store in = \springdamperlinethk,	
  spring location ratio/.store in = \springlocratio,	
  spring ratio/.store in = \springratio,	
  spring text side/.store in = \springtextside,	
  spring text/.store in = \springtext,	
  spring zigzag amplitude/.store in = \springzigzagamplitude,
  spring zigzag segment length/.store in = \springzigzagsegment,
  damper width ratio/.store in = \damperwidthratio,	
  damper height ratio/.store in = \damperheightratio,	
  damper piston size ratio/.store in = \damperpistonsizeratio,	
  damper piston position ratio/.store in = \damperpistonpositionratio,	
  damper text/.store in = \dampertext,	
  damper text side/.store in = \dampertextside,	
  wheel line thickness/.store in = \wheellinethk,	
  wheel radius/.store in = \wheelradius,	
  wheel location ratio/.store in = \wheellocratio,	
  wheel size ratio/.store in = \wheelsizeratio,	
  disp line location ratio/.store in = \displineineloc,	
  disp line/.store in = \displine,	
  disp arrow/.store in = \disparrow,	
  disp text/.store in = \disptext,	
  disp line thickness/.store in = \displinethk,	
  forcex location ratio/.store in = \forcelocratiox,	
  forcey location ratio/.store in = \forcelocratioy, 	
  force length/.store in = \forcelength,	
  force line thickness/.store in = \forcelinethk,	
  force text/.store in = \forcetext,	
  display displacement/.store in = \displaydisp,	
  display force/.store in = \displayforce,	
}

\newcommand{\sdofsys}[1][]{
\pgfkeys{/sdofsys, default, #1}
\tikzmath{
real \horsupplinethk, \vertsupplinethk, \springdamperlinethk, \masslinethk, \wheellinethk;
real \startcoordx, \startcoordy, \vertsuppshadethk, \horsuppshadethk, \suppwidth, \suppheight;
real \massheight, \masscoordx, \masswidth, \wheelradius, \wheellocratio, \wheelsizeratio, \springlocratio;
real \springratio, \damperwidthratio, \damperheightratio, \damperpistonsizeratio, \damperpistonpositionratio;
real \displine, \displineineloc, \forcelength, \forcelocratiox, \forcelocratioy, \forcelinethk;
real \tempx, \tempy;
real \dampersidenumber;
int \displaydisp, \displayforce;
coordinate \c, \cs, \cd, \cde, \cmass, \cw, \cdisp, \cforce;
%convert all units to pt
\vertsupplinethk=\vertsupplinethk;
\horsupplinethk=\horsupplinethk;
\springdamperlinethk=\springdamperlinethk;
\masslinethk=\masslinethk;
\wheellinethk=\wheellinethk;
\startcoordx=\startcoordx;
\startcoordy=\startcoordy;
\vertsuppshadethk=\vertsuppshadethk;
\horsuppshadethk=\horsuppshadethk;
\suppwidth=\suppwidth;
\suppheight=\suppheight;
\massheight=\massheight;
\masswidth=\masswidth;
\masscoordx=\masscoordx;
\wheelradius=\wheelradius;
\springlocratio=\springlocratio;
\displine=\displine;
\disparrow=\disparrow;
\displinethk=\displinethk;
\forcelength=\forcelength;
\forcelinethk=\forcelinethk;
\springzigzagamplitude=\springzigzagamplitude;
\springzigzagsegment=\springzigzagsegment;
% Support shade and line coordinates
\tempx = \startcoordx-\vertsuppshadethk;     \tempy = \startcoordy-\horsuppshadethk;      \c1 = (\tempx pt,\tempy pt);
\tempx = \startcoordx-\vertsuppshadethk;     \tempy = \startcoordy+\suppheight;    \c2 = (\tempx pt,\tempy pt);
\tempx = \startcoordx;         \tempy = \startcoordy+\suppheight;    \c3 = (\tempx pt,\tempy pt);
\tempx = \startcoordx;         \tempy = \startcoordy;          \c4 = (\tempx pt,\tempy pt);
\tempx = \startcoordx+\suppwidth;   \tempy = \startcoordy;          \c5 = (\tempx pt,\tempy pt);
\tempx = \startcoordx+\suppwidth;   \tempy = \startcoordy-\horsuppshadethk;      \c6 = (\tempx pt,\tempy pt);
% Mass Coordinates
\tempx = \startcoordx+\masscoordx;       \tempy = \startcoordy+\wheelradius*2;       \cmass1 = (\tempx pt,\tempy pt);
\tempx = \startcoordx+\masscoordx+\masswidth;  \tempy = \startcoordy+\wheelradius*2+\massheight;  \cmass2 = (\tempx pt,\tempy pt);
% Wheel Coordinates and Modified Wheel Radius
\tempx = \startcoordx+\masscoordx+\masswidth*\wheellocratio;        \tempy = \startcoordy+\wheelradius;     \cw1 = (\tempx pt,\tempy pt);  %Center of wheel 1
\tempx = \startcoordx+\masscoordx+\masswidth-\masswidth*\wheellocratio; \tempy = \startcoordy+\wheelradius;     \cw2 = (\tempx pt,\tempy pt);  %Center of wheel 2
\tempwr = \wheelradius-\wheellinethk;
\wheelradiusnew = \tempwr*\wheelsizeratio;   % new radius
% Spring Coordinates
\springaxisy = \startcoordy+\wheelradius*2+\massheight*(1-\springlocratio);  %spring y-axis;
\springlengthtotal = \masscoordx-\vertsupplinethk/2-\masslinethk/2;          %total length of the spring
\springlengthbegin = \springratio*\springlengthtotal;     %length of spring at the beginning
\springcoordx1 = \startcoordx+\vertsupplinethk/2;      %spring x-coordinates;
\springcoordx2 = \startcoordx+\springlengthtotal+\vertsupplinethk/2+\masslinethk/2;
\cs1 = (\springcoordx1 pt, \springaxisy pt);
\cs2 = (\springcoordx2 pt, \springaxisy pt);
% Damper Coordinates
\damperaxisy = \startcoordy+\wheelradius*2+\massheight*\springlocratio;    %damper y-axis;
\damperlengthtotal = \springlengthtotal;                              %total length of damper
\damperwidth = \damperwidthratio*\damperlengthtotal;               %damper width;
\damperheight = \damperheightratio*\massheight;           %damper height
\dx1 = \startcoordx+\vertsupplinethk/2;
\dx2 = \startcoordx+\vertsupplinethk/2+(\damperlengthtotal-\damperwidth)/2;
\dx3 = \dx2+\damperwidth*\damperpistonpositionratio;
\dx4 = \dx2+\damperwidth;
\dx5 = \startcoordx+\damperlengthtotal+\vertsupplinethk/2+\masslinethk/2;
\cd1 = (\dx1 pt,\damperaxisy pt);
\cd2 = (\dx2 pt,\damperaxisy pt);
\cd3 = (\dx4 pt,\damperaxisy+\damperheight/2 pt);
\cd4 = (\dx2 pt,\damperaxisy+\damperheight/2 pt);
\cd5 = (\dx2 pt,\damperaxisy-\damperheight/2 pt);
\cd6 = (\dx4 pt,\damperaxisy-\damperheight/2 pt);
\cd7 = (\dx3 pt,\damperaxisy+\damperheight*\damperpistonsizeratio/2 pt);
\cd8 = (\dx3 pt,\damperaxisy-\damperheight*\damperpistonsizeratio/2 pt);
\cd9 = (\dx3 pt,\damperaxisy pt);
\cde = (\dx5 pt,\damperaxisy pt);
% Displacement Axis Coordinates
\tempx = \startcoordx+\masscoordx+\masswidth*\displineineloc;  \tempy = \startcoordy+\wheelradius*2+\massheight;
\cdisp = (\tempx pt,\tempy pt);
%Force Arrow Coordinates
\tempx = \startcoordx+\masscoordx+\masswidth*\forcelocratiox;  \tempy = \startcoordy+\wheelradius*2+\massheight*\forcelocratioy;
\cforce = (\tempx pt,\tempy pt);
}
\begin{scope}[x=1pt, y=1pt, xshift = \startcoordx, yshift=\startcoordy, rotate=0];    % Draw everything in pt units
% Draw support shade and lines
\fill [gray] (\c1) -- (\c2) -- (\c3) -- (\c4) -- (\c5) -- (\c6) -- cycle;
\draw [line width = \horsupplinethk] (\c3) -- (\c4) [line width = \vertsupplinethk] -- (\c5);

% Draw Spring
\draw [line width = \springdamperlinethk, decorate,
     decoration={zigzag,pre=lineto, amplitude=\springzigzagamplitude pt, segment length=\springzigzagsegment pt,
     pre length=\springlengthbegin pt, post=lineto, post length=\springlengthbegin*0.90 pt}]
     (\cs1) -- node[\springtextside]{\springtext} (\cs2);

%Draw Damper
\draw [line width = \springdamperlinethk] (\cd1) -- (\cd2);
\draw [line width = \springdamperlinethk] (\cd9) -- (\cde);
\ifthenelse{\dampertextside=1}{
\draw [line width = \springdamperlinethk] (\cd3) -- node[above, inner sep=1pt]{\dampertext}(\cd4)  -- (\cd5)  -- (\cd6);
}
{
\draw [line width = \springdamperlinethk] (\cd3) -- (\cd4)  -- (\cd5)  -- node[below, inner sep=1pt]{\dampertext}(\cd6);
}
\draw [line width = \springdamperlinethk] (\cd7) -- (\cd8);

%Draw Mass
\draw [line width = \masslinethk] (\cmass1) rectangle (\cmass2) node[pos=0.5]{\masstext};

%Draw Wheels
\draw [line width = \wheellinethk](\cw1) circle [radius=\wheelradiusnew pt];
\draw [line width = \wheellinethk](\cw2) circle [radius=\wheelradiusnew pt];

\ifthenelse{\displaydisp=1}{
%Draw Displacement Vector
\draw [line width = \displinethk] (\cdisp) -- ++(0 cm, \displine pt);
\draw [line width = \displinethk] (\cdisp) ++(0 cm, \displine*0.8 pt) [->] -- node[above, yshift=-0.1 cm]{\disptext} ++(\disparrow pt, 0 cm);
}{}

\ifthenelse{\displayforce=1}{
%Draw Force Vector
\draw [line width = \forcelinethk] (\cforce) [->] -- node[above, yshift=-0.1 cm]{\forcetext} ++(\forcelength pt, 0 cm);
}{}
\end{scope}
}

%\begin{tikzpicture}
%\sdofsys[startx = 0cm, starty = 0cm,
%  support width = 4.0cm,
%  support height = 1.5cm,
%  vertical support line thickness = 1pt,
%  horizontal support line thickness = 1pt,
%  vertical support shade thickness = 0.3cm,
%  horizontal support shade thickness = 0.3cm,    
%  mass line thickness = 1pt,
%  mass height = 1cm,
%  mass width = 1.5cm,
%  massx = 1.5cm,
%  mass text = $m$,
%  spring and damper line thickness = 0.5pt,
%  spring location ratio = 0.25,
%  spring ratio = 0.15,
%  spring text = $k$,
%  spring text side = above,
%  damper width ratio = 0.15,
%  damper height ratio = 0.3,
%  damper piston size ratio = 0.7,
%  damper piston position ratio = 0.5,
%  damper text = $c$,
%  damper text side = 2,
%  wheel line thickness = 1pt,
%  wheel radius = 0.15cm,
%  wheel location ratio = 0.25,
%  wheel size ratio = 1,
%  disp line location ratio = 0.5,
%  disp line = 0.25cm,
%  disp arrow = 0.5cm,
%  disp text = $x(t)$,
%  disp line thickness = 0.5pt,
%  forcex location ratio = 1.0,
%  forcey location ratio = 0.5,  
%  force length = 1cm,
%  force line thickness = 0.5pt,
%  force text = $f(t)$,
%  display displacement = 1,
%  display force = 1]
%\end{tikzpicture}