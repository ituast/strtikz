\pgfkeys{
 /commonisolation/.is family, /commonisolation,
  default/.style = {
  startX = 0cm,
  startY = 0cm,
  support width = 1cm,
  support body = 5cm,
  support height = 0.15cm,
  support line thickness = 1.5pt,
  number of isolators = 4,
  isolator width = 0.3cm,
  isolator thickness = 0.2cm,
  isolator line thickness = 1.5pt,
  isolator drift=1cm,
  show isolation drift=0,
  isolator number to show drift=1,
  foundation thickness = 0.5cm,
  foundation side width = 1cm,
  mass text for slab = $m_\text{b}$,
  text location for mass=left,
  show mass text=1,
  arrow tip length=3pt,
  arrow tip width=2pt,
  isolation drift text=$x_\text{b}$,},
  startX/.store in = \startx,
  startY/.store in = \starty,
  support width/.store in = \supportsidewidth,
  support body/.store in = \supportbody,
  support height/.store in = \supportheight,
  support line thickness/.store in = \baselinet,
  number of isolators/.store in = \numberofisolators,
  isolator width/.store in = \isolationwidth,
  isolator thickness/.store in = \isolationdepth,
  isolator line thickness/.store in = \isolinet,
  isolator drift/.store in = \isolationdrift,
  show isolation drift/.store in = \showisodrift,
  isolator number to show drift/.store in = \isonumberdrift,
  foundation thickness/.store in = \foundationdepth,
  foundation side width/.store in = \foundsidew,
  mass text for slab/.store in = \masstextslab,
  text location for mass/.store in=\masstextloc,
  show mass text/.store in=\showmasstext,
  arrow tip length/.store in=\arrlen,
  arrow tip width/.store in=\arrwid,
  isolation drift text/.store in=\isodrifttext,
}

\newcommand{\commonisolation}[1][]{
\pgfkeys{/commonisolation, default, #1}
\tikzmath{
real \startx, \starty;
int \numberofisolators, \kiso, \showmasstext, \isonumberdrift;
real \supportsidewidth, \supportbody, \supportheight, \isolationwidth, \isolationdepth, \foundationdepth, \xiso, \isospace;
real \baselinet, \isolinet;
real \rigbasestartx, \rigbaseendx, \isoboty, \isotopy;
real \foundboty, \foundtopy, \foundstartx, \foundendx;
real \masstextoutersep, \masstextx, \masstexty;
real \isolationdrift, \isofounddrift, \arrlen, \arrwid;
%Conversion all units to the unit pt
\startx = \startx;
\starty = \starty;
\supportsidewidth = \supportsidewidth;
\supportbody =  \supportbody;
\supportheight = \supportheight;
\baselinet = \baselinet;
\isolationwidth = \isolationwidth;
\isolationdepth = \isolationdepth;
\isolinet = \isolinet;
\foundationdepth = \foundationdepth;
\foundsidew = \foundsidew;
\isolationdrift=\isolationdrift;
%End conversion, All values are now in pt units.
\rigbasestartx = -\supportsidewidth;
\rigbaseendx = \supportbody+\supportsidewidth;
\isoboty = -\supportheight-\baselinet/2-\isolationdepth;
\isotopy = -\supportheight-\baselinet/2;
if equal(\showisodrift,1) then {\isofounddrift=\isolationdrift;} else {\isofounddrift=0;};
\foundboty = -\supportheight-\baselinet-\isolationdepth-\foundationdepth;
\foundtopy = -\supportheight-\baselinet-\isolationdepth;
\foundstartx = -\foundsidew-\isofounddrift;
\foundendx = +\supportbody+\foundsidew-\isofounddrift;
%Some isolator properties
if equal(\numberofisolators,1) then{
\isospace = 0.0;
\xiso{1} = (\supportbody)/2.0;
}
else {
\isospace = (\supportbody)/(\numberofisolators-1);
for \kiso in {1,...,{\numberofisolators}}{\xiso{\kiso} = (\kiso-1)*\isospace;};
};
%%%% Mass text outer sep distance for mass, x and y coordinates;
\masstextoutersep = \supportsidewidth+\supportbody/2;
\masstextx = \supportbody/2;
\masstexty = -\supportheight/2;
}
\begin{scope}[x=1pt, y=1pt, xshift=\startx, yshift=\starty, rotate=0]; % Drawing everything in pt units
\tikzset{arrowbe/.style = {arrows={->[length=\arrlen, width=\arrwid]}}};
\tikzset{arrowben/.style = {arrows={<->[length=\arrlen, width=\arrwid]}}};

%%%%% Rigid Base
\fill [fill=gray, draw=black, line width = \baselinet] (\rigbasestartx, 0) rectangle (\rigbaseendx,-\supportheight);

%%%% Foundation
\fill [fill=gray]
(\foundstartx, \foundboty) rectangle (\foundendx,\foundtopy);
\draw [line width = \baselinet] (\foundstartx, \foundtopy) -- (\foundendx, \foundtopy);

\ifthenelse{\showmasstext=1}{
\node[inner sep=2pt, outer sep=\masstextoutersep, \masstextloc, opaque] at (\masstextx,\masstexty) {\masstextslab};}{}

%%%% Isolators
\foreach \j in {1,...,{\numberofisolators}}
{\ifthenelse{\showisodrift=1}{\filldraw [fill = white!80!black, draw=black, line width=\isolinet, line join=bevel]
(\xiso{\j}-\isolationdrift-\isolationwidth/2,\isoboty) -- ++(\isolationwidth,0)
-- ++(\isolationdrift,\isolationdepth) -- ++(-\isolationwidth,0) -- cycle;
\ifthenelse{\j=\isonumberdrift}{
\draw (\xiso{\j}-\isolationdrift,\isoboty+2pt) -- ++(0,-\foundationdepth-10pt);
\draw (\xiso{\j},\isotopy+2pt) -- ++(0,-\isolationdepth-\foundationdepth-10pt);
\draw [arrowbe] (\xiso{\j}-\isolationdrift,\isoboty-\foundationdepth-6pt) --
node[inner sep=2pt, anchor=north, midway]{\isodrifttext}  ++(\isolationdrift,0);}{}
} %%% If deflected shapes of isolators to be shown
{\filldraw [fill = white!20!black, draw=black, line width=\isolinet, line join=bevel]
(\xiso{\j},0) +(-\isolationwidth/2,\isoboty) rectangle +(\isolationwidth/2,\isotopy);} %%% If undeflected shapes of isolators to be shown.
}




\end{scope}
}

