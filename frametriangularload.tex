\pgfkeys{
 /frametriangularload/.is family, /frametriangularload,
  default/.style={
  startx = 0cm,
  starty = 0cm,
  endx = 5cm,
  endy = 5cm,
  number of arrows = 10,
  print first arrow = 1,
  height of arrows left= 0.5cm,
  height of arrows right= 0.5cm,
  space = 0.1cm,
  start ratio = 0.02,
  end ratio = 0.02,
  right load text = $\SI{10}{\kilo\newton/\meter}$,
  left load text = $\SI{10}{\kilo\newton/\meter}$,
  right text shiftx=0pt,
  right text shifty=0pt,
  left text shiftx=0pt,
  left text shifty=0pt,
  arrow length=5pt,
  arrow width=3pt,
  arrow scale=1,
  arrow line thickness=1pt},
  startx/.store in = \loadstartcoordx,
  starty/.store in = \loadstartcoordy,
  endx/.store in = \loadendcoordx,
  endy/.store in = \loadendcoordy,
  number of arrows/.store in = \loadarrownumber,
  print first arrow/.store in = \printfirstarrow,
  height of arrows left/.store in = \loadarrowheightleft,
  height of arrows right/.store in = \loadarrowheightright,
  space/.store in = \loadarrowspace,
  start ratio/.store in = \loadstartratio,
  end ratio/.store in = \loadendratio,
  right load text/.store in = \rightloadtext,
  left load text/.store in = \leftloadtext,
  right text shiftx/.store in = \righttextshiftx,
  right text shifty/.store in = \righttextshifty,
  left text shiftx/.store in = \lefttextshiftx,
  left text shifty/.store in = \lefttextshifty,
  arrow length/.store in = \arrlen,
  arrow width/.store in = \arrwid,
  arrow scale/.store in = \arrscale,
  arrow line thickness/.store in = \arrlinethk,}
%\loadstartcoordx: starting coordinate x
%\loadstartcoordy: starting coordinate y
%\loadendcoordx: end coordinate x
%\loadendcoordy: end coordinate y
%\loadarrownumber: number of arrows
%\loadarrowheightleft: height of arrows on the left hand side
%\loadarrowheightright: height of arrows on the right hand side
%\loadarrowspace: distance of arrows from the frame
%\loadstartratio: start(first) arrow start ratio
%\loadendratio: end (last) arrow start ratio
\newcommand{\frametriangularload}[1][]{
\pgfkeys{/frametriangularload, default, #1}
\tikzmath{
real \loadstartcoordx, \loadstartcoordy, \loadendcoordx, \loadendcoordy;
real \loadangle, \loadunx, \loaduny, \loaddxf, \loaddyf;
real \loaddxa, \loaddya, \loadaaa, \loadbbb;
real \loadxvalue, \loadyvalue;
real \loadwxvalue, \loadwyvalue, \loadarrowheightleft, \loadarrowheightright,
real \loadstartratio, \loadendratio, \lenf;
int \loadarrownumber, \loadarrownumbermo, \i, \j, \printfirstarrows;
real \rnarr, \loadarrowheight, \ldangle;
real \dah, \totalangle, \totallength;
real \arrlen, \arrwid, \arrscale, \arrlinethk;
real \righttextshiftx, \righttextshifty, \lefttextshiftx, \lefttextshiftx,,
%Conversion all units to the unit pt
\loadstartcoordx = \loadstartcoordx;
\loadstartcoordy = \loadstartcoordy;
\loadendcoordx = \loadendcoordx;
\loadendcoordy = \loadendcoordy;
\loadarrowheightleft = \loadarrowheightleft;
\loadarrowheightright = \loadarrowheightright;
\loadarrowspace = \loadarrowspace;
\arrlen=\arrlen;
\arrwid=\arrwid;
\arrlinethk=\arrlinethk;
\righttextshiftx=\righttextshiftx;
\righttextshifty=\righttextshifty;
\lefttextshiftx=\lefttextshiftx;
\lefttextshifty=\lefttextshifty;
%End conversion, All values are now in pt units.
\loaddxf = (\loadendcoordx-\loadstartcoordx); %dx for the whole frame
\loaddyf = (\loadendcoordy-\loadstartcoordy); %dy for the whole frame
\lenf = veclen(\loaddxf, \loaddyf);
\loadangle = atan2(\loaddyf,\loaddxf);
\loaddxa = \loaddxf*(1-(\loadstartratio+\loadendratio)); %dx for the start and end of arrow
\loaddya = \loaddyf*(1-(\loadstartratio+\loadendratio)); %dy for the start and end of arrow
\lena = veclen(\loaddxa, \loaddya);
\dah = \loadarrowheightright-\loadarrowheightleft;
\ldangle = atan2(\dah,\lena);
\totalangle = \loadangle + \ldangle;
\totallength = veclen(\lena, \dah);
\rnarr = \loadarrownumber;
\loadwxvalue = (\loaddxa)/(\rnarr-1); %dx for the single arrow
\loadwyvalue = (\loaddya)/(\rnarr-1); %dx for the single arrow
\loadxvalue{1} = \loadstartcoordx + \loaddxf*(\loadstartratio);
\loadyvalue{1} = \loadstartcoordy + \loaddyf*(\loadstartratio);
\loadxvalue{\loadarrownumber} = \loadendcoordx - \loaddxf*(\loadendratio);
\loadyvalue{\loadarrownumber} = \loadendcoordy - \loaddyf*(\loadendratio);
for \i in {2,...,{\loadarrownumber-1}}{
\loadxvalue{\i} = \loadxvalue{1} + (\i-1)*\loadwxvalue;
\loadyvalue{\i} = \loadyvalue{1} + (\i-1)*\loadwyvalue;
\loadarrowheight{\i} = \loadarrowheightleft + (\dah)/(\rnarr-1)*(\i-1);
};
\loadarrownumbermo = \loadarrownumber-1;
\loc = ifthenelse(\loadarrowheightleft>=0,"above","below");
% print {$\loaddxf$};
}
\tikzset{
forcearr/.style={
arrows={<[length=\arrlen, width=\arrwid, scale=\arrscale, flex=1, bend]-}  }
};

\tikzset{
arrforce/.style={
arrows={->[length=\arrlen, width=\arrwid, scale=\arrscale, flex=1, bend]}  }
};

\tikzset{
forcearrarr/.style={
arrows={<[length=\arrlen, width=\arrwid, scale=\arrscale, flex=1, bend]-
>[length=\arrlen, width=\arrwid, scale=\arrscale, flex=1, bend]}  }
};

\begin{scope}[x=1pt, y=1pt]; % Drawing everything in pt units
\foreach \i in {2,...,\loadarrownumbermo}{
\draw [forcearr, line width=\arrlinethk] (\loadxvalue{\i}, \loadyvalue{\i})
++(90+\loadangle:\loadarrowspace) --
++(90+\loadangle:\loadarrowheight{\i});}

\ifthenelse{\printfirstarrow=0}{\def \thearrow {arrforce} } {\def \thearrow {forcearrarr}};

\draw [\thearrow, line width=\arrlinethk] (\loadxvalue{1},\loadyvalue{1}) 
++(90+\loadangle:\loadarrowspace)--
node[\loc, at end, xshift=\lefttextshiftx, yshift=\lefttextshifty]{\leftloadtext}
++(90+\loadangle:\loadarrowheightleft) --
node[\loc, at end, xshift=\righttextshiftx, yshift=\righttextshifty]{\rightloadtext}
++(\totalangle:\totallength)-- ++(-90+\loadangle:\loadarrowheightright);
\end{scope}
}


