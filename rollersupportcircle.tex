\pgfkeys{
 /rollersupportcircle/.is family, /rollersupportcircle,
  default/.style = {x value = 0,
  y value = 0,
  line width = 1.5pt,
  base width = 0.6cm,
  base height = 0.2cm,
  circle radius = 3pt,
  circle line width = 1.0pt,},
  x value/.store in = \xvalue,
  y value/.store in = \yvalue,
  support width/.store in = \supwidth,
  line width/.store in = \linewidt,
  base width/.store in = \basewidth,
  base height/.store in = \baseheight,
  circle radius/.store in = \circrad,
  circle line width/.store in = \circlinewidt,
}

\newcommand{\rollersupportcircle}[1][]{
\pgfkeys{/rollersupportcircle, default, #1}
\tikzmath{
real \xvalue, \yvalue, \linewidt, \circrad;
real \basewidth, \baseheight, \circlinewidt, \circlocrat;
int \showcircles;
real \ccr;
%Conversion all units to the unit pt
\xvalue = \xvalue;
\yvalue = \yvalue;
\linewidt = \linewidt;
\basewidth = \basewidth;
\baseheight = \baseheight;
\circlinewidt = \circlinewidt;
\circrad = \circrad;
%All units are converted to pt
\ccr=0.90;
\circrad=\circrad-\circlinewidt/2;
%
\gfsx = -\basewidth/2; %gray fill start x
\gfex = \basewidth/2; %gray fill end x
\gfsy = -2*\circrad - \circlinewidt*\ccr - \linewidt/2; %gray fill start y
\gfey = -2*\circrad - \circlinewidt*\ccr -\baseheight ; %gray fill end y
%
\flsx = -\basewidth/2; % first line start x
\flex = \basewidth/2; % first line end x
\flsy = -2*\circrad - \linewidt/2-\circlinewidt*\ccr; % first line start y
\fley = \flsy; % first line end y
%
\mccx = 0; %middle circle center x
\mccy = -\circrad-\circlinewidt/2; %middle circle center y
%
}

\begin{scope}[x=1pt, y=1pt, xshift=\xvalue, yshift=\yvalue, rotate=0];    % Draw everything in pt units
	
	\draw [line width=\circlinewidt] (\mccx, \mccy) circle (\circrad);
	
	\fill [gray] (\gfsx, \gfsy) rectangle (\gfex, \gfey);
	\draw [line width=\linewidt] (\flsx, \flsy) -- (\flex, \fley);
%	\draw [line width=0.1pt] (0,0) -- ++(1cm,0);
%	\draw (0,5) -- ++(10,0);

\end{scope}
}