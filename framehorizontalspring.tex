\pgfkeys{
/framehorizontalspring/.is family, /framehorizontalspring,
default/.style={
startx = 0cm,
starty = 0cm,
endx = 5cm,
endy = 5cm,
number of springs = 10,
spring direction = 1,
space between springs = 0.02cm,
space = 0.1cm,
start ratio = 0.02,
end ratio = 0.02,
spring text = $\SI{10}{\kilo\newton/\meter}$,
text color = black,
text shiftx=0pt,
text shifty=0pt,
spring prelength ratio=0.25,
spring postlength ratio=0.25,
spring segment=0.15cm,
spring width=3pt,
spring scale=1,
spring line thickness=1pt,
spring color=black,
support width=0.3cm,
support depth=0.1cm,
support line thickness=1pt,
show support shade=1,
support shade color=gray,},
startx/.store in = \springstartcoordx,
starty/.store in = \springstartcoordy,
endx/.store in = \springendcoordx,
endy/.store in = \springendcoordy,
number of springs/.store in = \springnumber,
spring direction/.store in = \springdir,
space between springs/.store in = \spbtwspr,
space/.store in = \springspace,
start ratio/.store in = \springstartratio,
end ratio/.store in = \springendratio,
spring text/.store in = \springtext,
text color/.store in = \textcolor,
text shiftx/.store in = \textshiftx,
text shifty/.store in = \textshifty,
spring prelength ratio/.store in = \prelenratio,
spring postlength ratio/.store in = \postlenratio,
spring segment/.store in = \segm,
spring width/.store in = \sprwid,
spring scale/.store in = \sprscale,
spring line thickness/.store in = \sprlinethk,
spring color/.store in = \sprcolor,
support width/.store in = \suppwidth,
support depth/.store in = \suppdepth,
support line thickness/.store in = \supplinethk,
show support shade/.store in = \showsuppshade,
support shade color/.store in = \suppshadecol,}

\newcommand{\framehorizontalspring}[1][]{
\pgfkeys{/framehorizontalspring, default, #1}
\tikzmath{
real \springstartcoordx, \springstartcoordy, \springendcoordx, \springendcoordy;
real \springangle, \springunx, \springuny, \springdxf, \springdyf;
real \springdxa, \springdya, \springaaa, \springbbb;
real \springxvalue, \springyvalue;
real \springwxvalue, \springwyvalue;
real \springstartratio, \springendratio, \lenf;
real \textshiftx, \textshifty;
int \springnumber, \springnumbermo, \i, \j, \springdir;
real \sprlen, \sprwid, \sprscale, \sprlinethk;
real \prelenratio, \postlenratio, \prelength, \postlength, \segm;
real \spbtwspr;
real \spacebetweenspringsdx, \spacebetweenspringsdy;
real \totalspacebetweensprings, \totalspacebetweenspringsdx, \totalspacebetweenspringsdy;
real \supportdist, \supportdistdx, \supportdistdy, \supportendx, \supportendy;
real \suppwidth, \supdepth, \supplinethk, \suppangle;
int \showsuppshade;
%Conversion all units to the unit pt
\springstartcoordx = \springstartcoordx;
\springstartcoordy = \springstartcoordy;
\springendcoordx = \springendcoordx;
\springendcoordy = \springendcoordy;
\textshiftx=\textshiftx;
\textshifty=\textshifty;
\springspace = \springspace;
\sprwid=\sprwid;
\sprlinethk=\sprlinethk;
\segm=\segm;
\spbtwspr=\spbtwspr;
\suppdepth=\suppdepth;
\suppwidth=\suppwidth;
\supplinethk=\supplinethk;
%End conversion, All values are now in pt units.
%% Force application line
\springdxf = (\springendcoordx-\springstartcoordx); %dx for the whole frame
\springdyf = (\springendcoordy-\springstartcoordy); %dy for the whole frame
\springangle = atan2(\springdyf,\springdxf);
%% Start of springs
\sprforceratio= 1-(\springstartratio+\springendratio);
\springdxa = \springdxf*(\sprforceratio); %dx for the start and end of spring
\springdya = \springdyf*(\sprforceratio); %dy for the start and end of spring
%%
\spacebetweenspringsdx=\spbtwspr*cos(\springangle);
\spacebetweenspringsdy=\spbtwspr*sin(\springangle);
\totalspacebetweensprings=(\springnumber-1)*\spbtwspr;
\totalspacebetweenspringsdx=(\springnumber-1)*\spacebetweenspringsdx;
\totalspacebetweenspringsdy=(\springnumber-1)*\spacebetweenspringsdy;
%%
\suppdx=(\suppdepth+\supplinethk/2)*cos(\springangle);
\suppdy=(\suppdepth+\supplinethk/2)*sin(\springangle);
%%
\springwxvalue = (\springdxa-\totalspacebetweenspringsdx)/(\springnumber); %dx for the single spring
\springwyvalue = (\springdya-\totalspacebetweenspringsdy)/(\springnumber); %dx for the single spring
\horspringlen= veclen(\springwxvalue, \springwyvalue);
\sprlen = \horspringlen - \suppdepth - \supplinethk/2;
%
\prelength=\prelenratio*\sprlen;
\postlength=\postlenratio*\sprlen;
%%
\springxvalue{1} = \springstartcoordx + \springdxf*(\springstartratio) + \suppdx ;
\springyvalue{1} = \springstartcoordy + \springdyf*(\springstartratio) + \suppdy ;
%%
for \i in {2,...,\springnumber}{
\springxvalue{\i} = \springxvalue{1} + (\i-1)*(\springwxvalue + \spacebetweenspringsdx) ;
\springyvalue{\i} = \springyvalue{1} + (\i-1)*(\springwyvalue + \spacebetweenspringsdy);
};
%%%
\sprspang=90+\springangle;
%%
\texti=int(\springnumber/2);
\textshifty=\textshifty+\springspace;
%%
\supportdist=\springspace;
\supportdistdx=cos(90+\springangle)*\supportdist;
\supportdistdy=sin(90+\springangle)*\supportdist;
for \i in {1,...,{\springnumber}}{
	\supportendx{\i} = \springxvalue{\i} + \supportdistdx;
	\supportendy{\i} = \springyvalue{\i} + \supportdistdy;
};
\suppangle=\springangle-90;
}

%\node at (-1cm,1cm) {\horspringlen};
%\node at (-1cm,2cm) {\springdxf};
%\node at (-1cm,3cm) {\springdyf};


\begin{scope}[x=1pt, y=1pt]; % Drawing everything in pt units

%Springs
\foreach \i in {1,...,\springnumber}{
	\draw[line width=\sprlinethk, \sprcolor, line join=round]
	(\springxvalue{\i}, \springyvalue{\i}) ++(\sprspang:\springspace)
	decorate[
	decoration={zigzag,
		amplitude=\sprwid,
		pre length=\prelength,
		post length=\postlength,
		segment length=\segm,}] 
	{--	++(\springangle:\sprlen)} --++(\sprspang:-\springspace);
	
\fixedsupport[%
support width = \suppwidth,
support depth = \suppdepth,
line thickness = \supplinethk,
x value = \supportendx{\i},
y value = \supportendy{\i},
rotation = \suppangle,
show shade=\showsuppshade,
shade color=\suppshadecol,]
	
}

\path (\springstartcoordx, \springstartcoordy) --
node[sloped, above, xshift=\textshiftx, yshift=\textshifty]
{\springtext}(\springendcoordx, \springendcoordy);

\end{scope}
}