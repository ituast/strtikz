\pgfkeys{
 /frametrapezoidalload/.is family, /frametrapezoidalload,
  default/.style={
  startx = 0cm,
  starty = 0cm,
  endx = 5cm,
  endy = 5cm,
  number of arrows = 10,
  print first arrow = 1,
  print last arrow = 1,
  height of arrows left= 0.5cm,
  height of arrows mid = 1cm,
  height of arrows right= 0.5cm,
  mid ratio left = 0.4,
  mid ratio right = 0.3,
  space = 0.1cm,
  start ratio = 0.02,
  end ratio = 0.02,
  mid load text = $\SI{10}{\kilo\newton/\meter}$,
  left load text = $\SI{20}{\kilo\newton/\meter}$,
  right load text = $\SI{10}{\kilo\newton/\meter}$,
  text color = black,
  mid text shiftx=0pt,
  mid text shifty=0pt,
  left text shiftx=0pt,
  left text shifty=0pt,
  right text shiftx=0pt,
  right text shifty=0pt,
  arrow length=5pt,
  arrow width=3pt,
  arrow scale=1,
  arrow line thickness=1pt
  arrow color = black},
  startx/.store in = \loadstartcoordx,
  starty/.store in = \loadstartcoordy,
  endx/.store in = \loadendcoordx,
  endy/.store in = \loadendcoordy,
  number of arrows/.store in = \loadarrownumber,
  print first arrow/.store in = \printfirstarrow,
  print last arrow/.store in = \printlastarrow,
  height of arrows left/.store in = \loadarrowheightleft,
  height of arrows mid/.store in = \loadarrowheightmid,
  height of arrows right/.store in = \loadarrowheightright,
  mid ratio left/.store in = \midratioleft,
  mid ratio right/.store in = \midratioright,
  space/.store in = \loadarrowspace,
  start ratio/.store in = \loadstartratio,
  end ratio/.store in = \loadendratio,
  mid load text/.store in = \midloadtext,
  left load text/.store in = \leftloadtext,
  right load text/.store in = \rightloadtext,
  text color/.store in = \textcolor,
  mid text shiftx/.store in = \midtextshiftx,
  mid text shifty/.store in = \midtextshifty,
  left text shiftx/.store in = \lefttextshiftx,
  left text shifty/.store in = \lefttextshifty,
  right text shiftx/.store in = \righttextshiftx,
  right text shifty/.store in = \righttextshifty,
  arrow length/.store in = \arrlen,
  arrow width/.store in = \arrwid,
  arrow scale/.store in = \arrscale,
  arrow line thickness/.store in = \arrlinethk,
  arrow color/.store in = \arrcolor,}
%\loadstartcoordx: starting coordinate x
%\loadstartcoordy: starting coordinate y
%\loadendcoordx: end coordinate x
%\loadendcoordy: end coordinate y
%\loadarrownumber: number of arrows
%\loadarrowheightleft: height of arrows on the left hand side
%\loadarrowheightright: height of arrows on the right hand side
%\loadarrowspace: distance of arrows from the frame
%\loadstartratio: start(first) arrow start ratio
%\loadendratio: end (last) arrow start ratio
\newcommand{\frametrapezoidalload}[1][]{
\pgfkeys{/frametrapezoidalload, default, #1}
\tikzmath{
real \loadstartcoordx, \loadstartcoordy, \loadendcoordx, \loadendcoordy;
real \loadangle, \loadunx, \loaduny, \loaddxf, \loaddyf;
real \loaddxa, \loaddya, \loadaaa, \loadbbb;
real \loadxvalue, \loadyvalue;
real \loadwxvalue, \loadwyvalue;
real \loadarrowheightleft, \loadarrowheightright, \loadarrowheightmid;
real \midpointratio;
real \loadstartratio, \loadendratio, \lenf;
int \loadarrownumber, \loadarrownumbermo, \i, \j, \printfirstarrows, \leftendarrno;
real \loadarrowheight, \ldangleleft, \ldangleright;
real \dahleft, \dahright, \totalangleleft, \totallengthleft, \totallengthright;
real \arrlen, \arrwid, \arrscale, \arrlinethk;
real \midtextshiftx, \midtextshifty, \lefttextshiftx, \lefttextshiftx, \righttextshiftx, \righttextshiftx,
%Conversion all units to the unit pt
\loadstartcoordx = \loadstartcoordx;
\loadstartcoordy = \loadstartcoordy;
\loadendcoordx = \loadendcoordx;
\loadendcoordy = \loadendcoordy;
\loadarrowheightleft = \loadarrowheightleft;
\loadarrowheightmid = \loadarrowheightmid;
\loadarrowheightright = \loadarrowheightright;
\loadarrowspace = \loadarrowspace;
\arrlen=\arrlen;
\arrwid=\arrwid;
\arrlinethk=\arrlinethk;
\midtextshiftx=\midtextshiftx;
\midtextshifty=\midtextshifty;
\lefttextshiftx=\lefttextshiftx;
\lefttextshifty=\lefttextshifty;
\righttextshiftx=\righttextshiftx;
\righttextshifty=\righttextshifty;
%End conversion, All values are now in pt units.
%% Force application line
\loaddxf = (\loadendcoordx-\loadstartcoordx); %dx for the whole frame
\loaddyf = (\loadendcoordy-\loadstartcoordy); %dy for the whole frame
\lenf = veclen(\loaddxf, \loaddyf);
\loadangle = atan2(\loaddyf,\loaddxf);
\extraangle=0;
\loadanglearrows=\loadangle+\extraangle;
\loaddxfleft = \loaddxf*\midratioleft;
\loaddyfleft = \loaddyf*\midratioleft;
\lenfleft = \lenf*\midratioleft;
\loaddxfright = \loaddxf*\midratioright;
\loaddyfright = \loaddyf*\midratioright;
\lenfright = \lenf*\midratioright;
%% Start of arrows
\arrforceratio= 1-(\loadstartratio+\loadendratio);
\loaddxa = \loaddxf*(\arrforceratio); %dx for the start and end of arrow
\loaddya = \loaddyf*(\arrforceratio); %dy for the start and end of arrow
\lena = veclen(\loaddxa, \loaddya);
\loaddxaleft = \loaddxf*(\midratioleft-\loadstartratio); %dx for the start and end of arrow
\loaddyaleft = \loaddyf*(\midratioleft-\loadstartratio); %dy for the start and end of arrow
\lenaleft = \lenf*(\midratioleft-\loadstartratio);
\loaddxaright = \loaddxf*(\midratioright-\loadendratio)); %dx for the start and end of arrow
\loaddyaright = \loaddyf*(\midratioright-\loadendratio)); %dy for the start and end of arrow
\lenaright = \lenf*(\midratioright-\loadendratio);
\lenamid = \lena-\lenaleft-\lenaright;
%%%
\dahleft = \loadarrowheightmid-\loadarrowheightleft;
\dahright = \loadarrowheightmid-\loadarrowheightright;
%%%
\ldangleleft = atan2(\dahleft,\lenaleft);
\ldangleright = atan2(\dahright,\lenaright);
\totalangleleft = \loadangle + \ldangleleft;
\totalangleright = \loadangle + \ldangleright;
%%%
\totallengthleft = veclen(\lenaleft, \dahleft); %%length of inclined surface
\totallengthright = veclen(\lenaright, \dahright); %%length of inclined surface
%%
\loadwxvalue = (\loaddxa)/(\loadarrownumber-1); %dx for the single arrow
\loadwyvalue = (\loaddya)/(\loadarrownumber-1); %dx for the single arrow
\loadxvalue{1} = \loadstartcoordx + \loaddxf*(\loadstartratio);
\loadyvalue{1} = \loadstartcoordy + \loaddyf*(\loadstartratio);
\loadxvalue{\loadarrownumber} = \loadxvalue{1} + \loaddxa;
\loadyvalue{\loadarrownumber} = \loadyvalue{1} + \loaddya;
%%
\loadarrownumbermo = \loadarrownumber-1;
%%
\arrlenratioleft = (\midratioleft-\loadstartratio) / (\arrforceratio);
\arrlenratioright = (\midratioright-\loadendratio) / (\arrforceratio);
\leftstartarrno=2;
\leftendarrno = int((\loadarrownumber-1)*\arrlenratioleft)+1;
%\rightstartarrno = int(\leftendarrno + 1);
\rightstartarrno = \loadarrownumber-int((\loadarrownumber-1)*\arrlenratioright);
\rightendarrno=\loadarrownumbermo;
%%
\leftendarrratio=(\leftendarrno-1)*\loadwxvalue/\loaddxaleft;
\rightstartarrratio=(\loadarrownumber-\rightstartarrno)*\loadwxvalue/\loaddxaright;
\leftendarrdist=\leftendarrratio*\lenaleft;
\dahleftend=\leftendarrratio*\dahleft;
\rightstartarrdist=\rightstartarrratio*\lenaright;
\dahrightstart=\rightstartarrratio*\dahright;
%% Left Region
for \i in {2,...,{\leftendarrno}}{
\loadxvalue{\i} = \loadxvalue{1} + (\i-1)*\loadwxvalue;
\loadyvalue{\i} = \loadyvalue{1} + (\i-1)*\loadwyvalue;
\loadarrowheight{\i} = \loadarrowheightleft + (\dahleftend)/(\leftendarrno-1)*(\i-1);
};
%% Mid Region
for \i in {{\leftendarrno+1},...,{\rightstartarrno-1}}{
	\loadxvalue{\i} = \loadxvalue{1} + (\i-1)*\loadwxvalue;
	\loadyvalue{\i} = \loadyvalue{1} + (\i-1)*\loadwyvalue;
	\loadarrowheight{\i} = \loadarrowheightmid;
};
%% Right Region
for \i in {{\rightendarrno},...,{\rightstartarrno}}{
\loadxvalue{\i} = \loadxvalue{\loadarrownumber} - (\loadarrownumber - (\i))*\loadwxvalue;
\loadyvalue{\i} = \loadyvalue{\loadarrownumber} - (\loadarrownumber - (\i))*\loadwyvalue;
\loadarrowheight{\i} = \loadarrowheightright + (\dahrightstart)/(\loadarrownumber-\rightstartarrno)*(\loadarrownumber - (\i));
};
\loc = ifthenelse(\loadarrowheightleft>=0,"above","below");
% print {$\loaddxf$};
}


\tikzset{
forcearr/.style={
arrows={<[length=\arrlen, width=\arrwid, scale=\arrscale, flex=1, bend]-}  }
};

\tikzset{
arrforce/.style={
arrows={->[length=\arrlen, width=\arrwid, scale=\arrscale, flex=1, bend]}  }
};

\tikzset{
forcearrarr/.style={
arrows={<[length=\arrlen, width=\arrwid, scale=\arrscale, flex=1, bend]-
>[length=\arrlen, width=\arrwid, scale=\arrscale, flex=1, bend]}  }
};

\begin{scope}[x=1pt, y=1pt]; % Drawing everything in pt units
\foreach \i in {2,...,\loadarrownumbermo}{
\draw [forcearr, line width=\arrlinethk, \arrcolor]
(\loadxvalue{\i}, \loadyvalue{\i})
++(90+\loadanglearrows:\loadarrowspace) --
++(90+\loadanglearrows:\loadarrowheight{\i});}

\ifthenelse{\printfirstarrow=0}
{
	\ifthenelse{\printlastarrow=0}{\def \thearrow {}}{\def \thearrow {arrforce}};
}
{
	\ifthenelse{\printlastarrow=0}{\def \thearrow {forcearr}}{\def \thearrow {forcearrarr}};
};


\draw [\thearrow, line width=\arrlinethk, \arrcolor]
(\loadxvalue{1},\loadyvalue{1}) 
++(90+\loadangle:\loadarrowspace)--
node[\loc, at end, xshift=\lefttextshiftx, yshift=\lefttextshifty, \textcolor]{\leftloadtext}
++(90+\loadangle+\extraangle:\loadarrowheightleft) --
++(\totalangleleft:\totallengthleft)--
node[\loc, xshift=\midtextshiftx, yshift=\midtextshifty, \textcolor]{\midloadtext}
++(\loadangle:\lenamid) --
node[\loc, at end, xshift=\righttextshiftx, yshift=\righttextshifty, \textcolor]{\rightloadtext}
++(-\ldangleright+\loadangle:\totallengthright)--
++(-90+\loadanglearrows:\loadarrowheightright);
\end{scope}
}