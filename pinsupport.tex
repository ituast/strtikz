\pgfkeys{
 /pinsupport/.is family, /pinsupport,
  default/.style = {triangle width = 0.5cm,
  line width = 1cm,
  line depth = 0.2cm,
  line thickness = 2pt,
  radius ratio = 0.2,
  x value = 0,
  y value = 0,
  show circle = 1,},
  triangle width/.store in = \triw,
  line width/.store in = \linew,
  line depth/.store in = \lined,
  line thickness/.store in = \linet,
  radius ratio/.store in = \radratio,
  x value/.store in = \xvalue,
  y value/.store in = \yvalue,
  show circle/.store in = \showcircle,
}

\newcommand{\pinsupport}[1][]{
\pgfkeys{/pinsupport, default, #1}
\tikzmath{
real \triw, \linew, \lined, \xvalue, \yvalue, \radratio;
real \rad;
int showcircle;
%Conversion all units to the unit pt
\triw  = \triw;
\linew = \linew;
\lined = \lined;
\xvalue = \xvalue;
\yvalue = \yvalue;
\radratio = \radratio;
\rad = \radratio*\triw;
\c1 = (0, 0);
\c2 = (-\triw/2, -\triw);
\c3 = (+\triw/2, -\triw);
\c4 = (-\linew/2, -\triw);
\c5 = (+\linew/2, -\triw);
\c6 = (-\linew/2, -\triw-\lined);
\c7 = (+\linew/2, -\triw-\lined);
}
\begin{scope}[x=1pt, y=1pt, xshift=\xvalue, yshift=\yvalue, rotate=0];    % Draw everything in pt units
\draw [line join=bevel, line width = \linet] (0, 0) -- (-\triw/2, -\triw) -- (+\triw/2, -\triw) -- cycle;
\ifthenelse{\showcircle=1}
	{\filldraw [fill=white, line width = \linet] (0, 0) circle[radius=\rad];}
	{;}
\fill [gray] (-\linew/2, -\triw-\lined) rectangle (+\linew/2, -\triw);
\draw [line width = \linet] (-\linew/2, -\triw) -- (+\linew/2, -\triw);
\end{scope}
}