\pgfkeys{
 /framedistributedload/.is family, /framedistributedload,
  default/.style={
  startx = 0cm,
  starty = 0cm,
  endx = 5cm,
  endy = 5cm,
  number of arrows = 10,
  height of arrows = 0.5cm,
  space = 0.1cm,
  start ratio = 0.02,
  end ratio = 0.02,
  load text = $\SI{10}{\kilo\newton/\meter}$,
  text shiftx=0pt,
  text shifty=0pt,
  arrow length=5pt,
  arrow width=3pt,
  arrow scale=1,
  arrow line thickness=1pt,  
  },
  startx/.store in = \loadstartcoordx,
  starty/.store in = \loadstartcoordy,
  endx/.store in = \loadendcoordx,
  endy/.store in = \loadendcoordy,
  number of arrows/.store in = \loadarrownumber,
  height of arrows/.store in = \loadarrowheight,
  space/.store in = \loadarrowspace,
  start ratio/.store in = \loadstartratio,
  end ratio/.store in = \loadendratio,
  load text/.store in = \loadtext,
text shiftx/.store in = \textshiftx,
text shifty/.store in = \textshifty,
arrow length/.store in = \arrlen,
arrow width/.store in = \arrwid,
arrow scale/.store in = \arrscale,
arrow line thickness/.store in = \arrlinethk,  
}

%\loadstartcoordx: starting coordinate x
%\loadstartcoordy: starting coordinate y
%\loadendcoordx: end coordinate x
%\loadendcoordy: end coordinate y
%\loadarrownumber: number of arrows
%\loadarrowheight: height of arrows
%\loadarrowspace: distance of arrows from the frame
%\loadstartratio: start(first) arrow start ratio
%\loadendratio: end (last) arrow start ratio
\newcommand{\framedistributedload}[1][]{
\pgfkeys{/framedistributedload, default, #1}
\tikzmath{
real \loadstartcoordx, \loadstartcoordy, \loadendcoordx, \loadendcoordy;
real \loadangle, \loadunx, \loaduny, \loaddxf, \loaddyf, \loaddxa, \loaddya, \loadaaa, \loadbbb;
real \loadxvalue, \loadyvalue;
real \loadwxvalue, \loadwyvalue, \loadarrowheight, \loadstartratio, \loadendratio, \lenf;
int \loadarrownumber, \loadarrownumbermo, \i, \j;
real \rnarr;
real \arrlen, \arrwid, \arrscale, \arrlinethk;
real \textshiftx, \textshifty;
%Conversion all units to the unit pt
\loadstartcoordx = \loadstartcoordx;
\loadstartcoordy = \loadstartcoordy;
\loadendcoordx = \loadendcoordx;
\loadendcoordy = \loadendcoordy;
\loadarrowheight = \loadarrowheight;
\loadarrowspace = \loadarrowspace;
\arrlen=\arrlen;
\arrwid=\arrwid;
\arrlinethk=\arrlinethk;
\textshiftx=\textshiftx;
\textshifty=\textshifty;
%End conversion, All values are now in pt units.
 \loaddxf = (\loadendcoordx-\loadstartcoordx); %dx for the whole frame
 \loaddyf = (\loadendcoordy-\loadstartcoordy); %dy for the whole frame
 \lenf = veclen(\loaddxf, \loaddyf);
 \loadangle = atan2(\loaddyf,\loaddxf);
 \loaddxa = \loaddxf*(1-(\loadstartratio+\loadendratio)); %dx for the start and end of arrow
 \loaddya = \loaddyf*(1-(\loadstartratio+\loadendratio)); %dy for the start and end of arrow
 \lena = veclen(\loaddxa, \loaddya);
 \rnarr = \loadarrownumber;
 \loadwxvalue = (\loaddxa)/(\rnarr-1); %dx for the single arrow
 \loadwyvalue = (\loaddya)/(\rnarr-1); %dx for the single arrow
 \loadxvalue{1} = \loadstartcoordx + \loaddxf*(\loadstartratio);
 \loadyvalue{1} = \loadstartcoordy + \loaddyf*(\loadstartratio);
 \loadxvalue{\loadarrownumber} = \loadendcoordx - \loaddxf*(\loadendratio);
 \loadyvalue{\loadarrownumber} = \loadendcoordy - \loaddyf*(\loadendratio);
 for \i in {2,...,{\loadarrownumber-1}}{
 \loadxvalue{\i} = \loadxvalue{1} + (\i-1)*\loadwxvalue;
 \loadyvalue{\i} = \loadyvalue{1} + (\i-1)*\loadwyvalue;
 };
 \loadarrownumbermo = \loadarrownumber-1;
 \loc = ifthenelse(\loadarrowheight>=0,"above","below");
% print {$\loaddxf$};
}

\tikzset{
	forcearr/.style={
		arrows={<[length=\arrlen, width=\arrwid, scale=\arrscale, flex=1, bend]-}  }
};

\tikzset{
	arrforce/.style={
		arrows={->[length=\arrlen, width=\arrwid, scale=\arrscale, flex=1, bend]}  }
};

\tikzset{
	forcearrarr/.style={
		arrows={<[length=\arrlen, width=\arrwid, scale=\arrscale, flex=1, bend]-
			>[length=\arrlen, width=\arrwid, scale=\arrscale, flex=1, bend]}  }
};

\begin{scope}[x=1pt, y=1pt]; % Drawing everything in pt units
\foreach \i in {2,...,\loadarrownumbermo}{

\draw [forcearr, line width=\arrlinethk]
(\loadxvalue{\i}, \loadyvalue{\i})
++(90+\loadangle:\loadarrowspace) --
++(90+\loadangle:\loadarrowheight);}

\draw [forcearrarr, line width=\arrlinethk]
(\loadxvalue{1},\loadyvalue{1}) ++(90+\loadangle:\loadarrowspace)--
++(90+\loadangle:\loadarrowheight) --
node[\loc,midway,sloped, xshift=\textshiftx, yshift=\textshifty]{\loadtext}
++(\loadangle:\lena)-- ++(-90+\loadangle:\loadarrowheight);
\end{scope}
}

