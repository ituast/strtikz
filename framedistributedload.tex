\pgfkeys{
 /framedistributedload/.is family, /framedistributedload,
  default/.style={
  startx = 0cm,
  starty = 0cm,
  endx = 5cm,
  endy = 5cm,
  number of arrows = 10,
  height of arrows = 0.5cm,
  space = 0.1cm,
  start ratio = 0.02,
  end ratio = 0.02,
  load text = $\SI{10}{\kilo\newton/\meter}$},
  startx/.store in = \loadstartcoordx,
  starty/.store in = \loadstartcoordy,
  endx/.store in = \loadendcoordx,
  endy/.store in = \loadendcoordy,
  number of arrows/.store in = \loadarrownumber,
  height of arrows/.store in = \loadarrowheight,
  space/.store in = \loadarrowspace,
  start ratio/.store in = \loadstartratio,
  end ratio/.store in = \loadendratio,
  load text/.store in = \loadtext,
}

%\loadstartcoordx: starting coordinate x
%\loadstartcoordy: starting coordinate y
%\loadendcoordx: end coordinate x
%\loadendcoordy: end coordinate y
%\loadarrownumber: number of arrows
%\loadarrowheight: height of arrows
%\loadarrowspace: distance of arrows from the frame
%\loadstartratio: start(first) arrow start ratio
%\loadendratio: end (last) arrow start ratio
\newcommand{\framedistributedload}[1][]{
\pgfkeys{/framedistributedload, default, #1}
\tikzmath{
real \loadstartcoordx, \loadstartcoordy, \loadendcoordx, \loadendcoordy;
real \loadangle, \loadunx, \loaduny, \loaddxf, \loaddyf, \loaddxa, \loaddya, \loadaaa, \loadbbb;
real \loadxvalue, \loadyvalue;
real \loadwxvalue, \loadwyvalue, \loadarrowheight, \loadstartratio, \loadendratio, \lenf;
int \loadarrownumber, \loadarrownumbermo, \i, \j;
real \rnarr;
%Conversion all units to the unit pt
\loadstartcoordx = \loadstartcoordx;
\loadstartcoordy = \loadstartcoordy;
\loadendcoordx = \loadendcoordx;
\loadendcoordy = \loadendcoordy;
\loadarrowheight = \loadarrowheight;
\loadarrowspace = \loadarrowspace;
%End conversion, All values are now in pt units.
 \loaddxf = (\loadendcoordx-\loadstartcoordx); %dx for the whole frame
 \loaddyf = (\loadendcoordy-\loadstartcoordy); %dy for the whole frame
 \lenf = veclen(\loaddxf, \loaddyf);
 \loadangle = atan2(\loaddyf,\loaddxf);
 \loaddxa = \loaddxf*(1-(\loadstartratio+\loadendratio)); %dx for the start and end of arrow
 \loaddya = \loaddyf*(1-(\loadstartratio+\loadendratio)); %dy for the start and end of arrow
 \lena = veclen(\loaddxa, \loaddya);
 \rnarr = \loadarrownumber;
 \loadwxvalue = (\loaddxa)/(\rnarr-1); %dx for the single arrow
 \loadwyvalue = (\loaddya)/(\rnarr-1); %dx for the single arrow
 \loadxvalue{1} = \loadstartcoordx + \loaddxf*(\loadstartratio);
 \loadyvalue{1} = \loadstartcoordy + \loaddyf*(\loadstartratio);
 \loadxvalue{\loadarrownumber} = \loadendcoordx - \loaddxf*(\loadendratio);
 \loadyvalue{\loadarrownumber} = \loadendcoordy - \loaddyf*(\loadendratio);
 for \i in {2,...,{\loadarrownumber-1}}{
 \loadxvalue{\i} = \loadxvalue{1} + (\i-1)*\loadwxvalue;
 \loadyvalue{\i} = \loadyvalue{1} + (\i-1)*\loadwyvalue;
 };
 \loadarrownumbermo = \loadarrownumber-1;
 \loc = ifthenelse(\loadarrowheight>0,"above","below");
% print {$\loaddxf$};
}
\begin{scope}[x=1pt, y=1pt]; % Drawing everything in pt units
\foreach \i in {2,...,\loadarrownumbermo}{
\draw [<-, line width=0.5] (\loadxvalue{\i}, \loadyvalue{\i}) ++(90+\loadangle:\loadarrowspace) -- ++(90+\loadangle:\loadarrowheight);}
\draw [<->, line width=0.5] (\loadxvalue{1},\loadyvalue{1}) ++(90+\loadangle:\loadarrowspace)--
++(90+\loadangle:\loadarrowheight) -- node[\loc,midway,sloped]{\loadtext}
++(\loadangle:\lena)-- ++(-90+\loadangle:\loadarrowheight);
\end{scope}
}

%\begin{tikzpicture}
%\draw [line width=1](0 cm, 0 cm) -- (5 cm, 5 cm);
%\framedistributedload[startx = 0cm, starty = 0cm,
%endx = 5cm, endy = 5cm,
%number of arrows = 10,
%height of arrows = 0.5cm,
%space = 0.1cm,
%start ratio = 0.02,
%end ratio = 0.02,
%load text = $\SI{10}{\kilo\newton/\meter}$]
%\end{tikzpicture}

